\section{Projectopdracht}
Om een beschrijving te geven van de taken die we gaan doen in dit project, geven we eerst informatie over onze opdrachtgever en wat zijn belang bij het project is. Daarna geven we de probleemstelling, gevolgd door de beschrijving van de opdracht en onze doelstelling. Aan de hand hiervan stellen we de op te leveren producten en diensten op, evenals de beperkingen en voorwaarden die gelden voor de opdracht.

\subsection{Opdrachtgever}
De opdrachtgever voor dit project bestaat uit NedTrain en de Technische Universiteit Delft. De contactpersoon binnen NedTrain is ir. Bob Huisman (\href{mailto:b.huisman@nedtrain.nl}{b.huisman@nedtrain.nl}). Hij heeft ook vorige projecten die zich bezig hielden met software voor NedTrain begeleid. Onze begeleider van de TU Delft is prof. dr. Cees Witteveen (\href{mailto:c.witteveen@tudelft.nl}{c.witteveen@tudelft.nl}). Hij is werkzaam binnen de Algoritmiek Groep op de faculteit EWI (Elektrotechniek, Wiskunde en Informatica).

\subsection{Probleemstelling}
De NedTrain planner, ontwikkeld in samenwerking met de TU Delft, is een grafische tool die moeilijke schedulingsproblemen kan visualiseren. Met behulp van zelfgemaakte oplossing-scripts kunnen deze problemen opgelost worden. De tool geeft vervolgens de mogelijkheid om deze oplossingen te bekijken en aanpassingen te maken om zo de veranderingen te onderzoeken. Deze planner wordt gebruikt door NedTrain om onderzoek te doen naar schedulingsalgoritmen en het verbeteren daarvan. Daarnaast heeft het voor de TU Delft ook een educatief doel, doordat deze op een visuele manier duidelijke maakt hoe schedulingsalgoritmen te werk gaan. De huidige versie is een resultaat van meerdere projecten van studenten aan de TU Delft. De applicatie dient als proof of concept om te laten zien wat er mogelijk is met complexe algoritmen op het gebied van het plannen van treinonderhoud. NedTrain zou graag willen dat de laatste ontwikkelingen op het gebied van schedulingsalgoritmen ook in de tool worden ge\"implementeerd. Hierdoor kan NedTrain beter inzicht krijgen in wat voor ontwikkelingen de laatste jaren gemaakt zijn en of ze gebruikt kunnen worden voor andere toekomstige systemen.

\subsection{Doelstelling Project}
Voor het berekenen en inzicht krijgen in de flexibiliteit van schedulingsproblemen bestaat er nog niet zoveel functionaliteit voor de tool. Daarnaast zijn er in de laatste jaren op dit gebied ontwikkelingen gemaakt die goed gebruikt zouden kunnen door NedTrain voor toekomstige systemen. Aangezien er ook op de TU Delft gewerkt wordt aan dit onderwerp is er daar ook interesse voor het implementeren van deze missende functionaliteit, omdat dit tot nieuwe inzichten kan leiden en de tool gebruikt kan worden als educatief hulpmiddel. Op dit moment kunnen er bij het verschuiven van de starttijden van taken resource conflicten ontstaan wat leidt tot een niet mogelijke rooster. Om dit op te lossen zou de NedTrain planner uitgebreid kunnen worden met een algoritme dat de chaining methode implementeert zodat de taken verschoven kunnen worden zonder dat er een resource conflict ontstaat \cite{seminarium2014}. Er worden hier namelijk zogenaamde chains, oftewel kettingen, van taken gemaakt waardoor er geforceerd wordt dat taken in dezelfde volgorde uitgevoerd blijven worden en dus niet de capaciteit van de resources overschrijden. Het verschuiven van taken kan echter wel het gevolg hebben dat ook ander taken verschoven worden. Om dit tegen te gaan, wil NedTrain ook dat er door middel van een \emph{Linear Programming (LP)} solver een schema wordt gevonden, waarbij elke taak een interval heeft waarover deze verschoven kan worden, zonder dat dit invloed heeft op de andere taken. De mate waarin dit kan gebeuren, wordt gezien als de flexibiliteit van een schema. Visueel zullen er daardoor dus ook aanpassingen gemaakt moeten worden, omdat de huidige tool nog geen manier heeft om de maat van de flexibiliteit van een schema te tonen. Hierbij moet er wel rekening worden gehouden dat oude solvers die niet de flexibiliteit van schemas berekenen nog steeds werken op de tool. Ook moet de NedTrain planner voor educatieve doeleinden gebruikt kunnen worden en dus inzicht kunnen bieden in hoe deze solvers aan het schema gekomen is. Tenslotte moet er gekeken worden of de tool, die op dit moment alleen onder Linux werkt, geport kan worden naar Windows 7 om de bruikbaarheid te vergroten.

\subsection{Opdrachtformulering}
\label{subsec:opdrachtformulering}
De volgende hoofdzaken zullen behoren tot de verantwoordelijkheden van de projectgroep:
\begin{enumerate}
	\item \label{enum:chaining} Het implementeren van een solver die gebruik maakt van de chaining methode.
	\item \label{enum:LP} Het uitbreiden van de solver met de functionaliteit om aan elke taak een onafhankelijk interval toewijst waarover het zonder andere gevolgen verschoven kan worden, door middel van een LP solver.
	\item \label{enum:visueel} Het visualiseren van de manier waarop de solver tot zijn antwoord komt.
	\item \label{enum:windows} Onderzoek doen naar de hoeveelheid tijd die nodig is om de NedTrain planner ook beschikbaar te maken voor Windows 7 en de nieuwst Qt versie. Als verwacht wordt dat het porten in redelijke tijd uitgevoerd kan worden, zal dit ook gedaan worden.
	\item \label{enum:gebruiker} De gebruikerservaring verbeteren door een aantal opties en shortcuts toe te voegen en door een aantal functies te veranderen.
\end{enumerate}

\subsection{Op te leveren Producten en Diensten}
\label{subsec:producten}
De volgende features worden verwacht om ge\"implementeerd te worden. Deze zijn onderverdeeld in verschillende maten van prioriteit en horen bij \'e\'en van de hoofdzaken genoemd in \ref{subsec:opdrachtformulering}. De nummers van de onderstaande producten en diensten komen overeen met een nummer van \'e\'en van de hoofdzaken in \ref{subsec:opdrachtformulering}

\subsubsection*{Hoge Prioriteit}
\begin{itemize}
	\item[\ref{enum:chaining}.] Het maken van een solver die er voor zorgt dat er bij het verslepen van een taak geen resource conflicten ontstaan, door middel van de chaining methode.
	\item[\ref{enum:LP}.] Het toevoegen van functionaliteit aan de solver, zodat deze aan elke taak een onafhankelijk interval toewijst waarover deze taak verschoven kan worden, door middel van een LP solver.
	\item[\ref{enum:visueel}.] Het visualiseren van chains in de interface van taken, zodat duidelijk is tot welke chain elke taak behoort en hoe het algoritme hiertoe gekomen is.
	\item[\ref{enum:visueel}.] Het visualiseren van de flexibiliteit van elke taak en van het schema in het geheel door het tonen van intervallen en het geven van de maat van de flexibiliteit.
	\item[\ref{enum:windows}.] De tool voor zowel Linux als Windows 7 beschikbaar maken. Op dit moment is deze alleen voor Linux systemen beschikbaar.
\end{itemize}

\subsubsection*{Gemiddelde Prioriteit}
\begin{itemize}
	\item[\ref{enum:visueel}.] Nieuwe view voor resources, waarbij weergegeven wordt tot welke chains de gebruikte resources behoren.
	\item[\ref{enum:windows}.] Het porten van de applicatie van Qt versie 4.8 naar versie 5.2.
	\item[\ref{enum:gebruiker}.] Undo en redo functies om het verslepen en veranderen van taken ongedaan te maken. De laatste tien acties worden opgeslagen, zodat die ongedaan gemaakt kunnen worden. Deze acties zijn niet meer ongedaan te maken na het afsluiten van de tool.
\end{itemize}

\subsubsection*{Lage Prioriteit}
\begin{itemize}
	\item[\ref{enum:visueel}.] Het verduidelijken van het vergelijken van verschillende oplossingen.
	\item[\ref{enum:gebruiker}.] Horizontaal scrollen moet mogelijk worden met shift\plus scroll. Deze functie heeft verder hetzelfde gedrag als wanneer er met de muis een scrollbalk verschoven wordt. De muis moet in het gebied staan dat horizontaal gescrolld moet worden. 
	\item[\ref{enum:gebruiker}.] Zoomfunctie moet ook mogelijk zijn met ctrl\plus scroll. Dit werkt verder hetzelfde als de knoppen 'inzoomen' en 'uitzoomen'.
	\item[\ref{enum:gebruiker}.] Sluitknopje op de tab van elke instantie die geopend is in de tool.
\end{itemize}

\subsection{Eisen en Beperkingen}
Er wordt ge\"eisd dat alle in \ref{subsec:producten} genoemde features met hoge prioriteit, met uitzondering van het porten naar Windows 7, worden uitgevoerd. Het porten naar Windows 7 zal onderzocht worden en worden uitgevoerd als verwacht wordt dat dit in redelijke tijd kan gebeuren. Eventuele veranderingen aan de eisen zullen met de opdrachtgever gecommuniceerd worden. Daarnaast is kwaliteit van code en documentatie ook een belangrijke eis van de opdrachtgever, zodat het uiteindelijke product makkelijk gebruikt, uitgebreid en overgedragen kan worden.

\subsection{Voorwaarden}
Van de projectleden wordt verwacht om duidelijk en op tijd te communiceren met de opdrachtgever, bijvoorbeeld in het geval van tegenvallers bij de implementatie of bij het maken van een afspraak voor feedback. Bij eventuele wijzigingen aan eisen en dergelijke moet dit zo spoedig mogelijk met de projectleden besproken worden. Van de opdrachtgever wordt verwacht dat deze duidelijk de eisen van het product communiceert en feedback geeft op de geleverde documentatie.
