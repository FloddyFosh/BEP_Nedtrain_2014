\section{Projectopdracht}
\subsection{Projectomgeving}
Onze opdrachtgever is NedTrain, het bedrijf dat zorgt voor het onderhoud aan en de reparatie van de treinen van de Nederlandse Spoorwegen (NS). NedTrain en TU Delft ontwikkelen samen een tool voor het maken van schema's voor het repareren van de treinen. 
De huidige situatie. 
\subsection{Doelstelling project}
De opdrachtgever wil een tool hebben waarmee de reparatie van treinen geplant kan worden. Ook moet deze tool inzicht bieden hoe deze aan het schema komt. 
Er zijn een aantal doelen die we willen behalen.
\begin{itemize}
	\item Onderzoek doen naar de hoeveelheid tijd die nodig is om deze tool ook beschikbaar te maken voor Windows 7. Als verwacht wordt dat het poorten naar Windows niet te lang duurt, dan gaan we dit ook uitvoeren. 
	\item De flexibilteit van het schema moet zichtbaar zijn door te laten zien hoeveel bepaalde taken mogen uitlopen. 
	\item De gebruiker van de tool moet taken kunnen verplaatsen en kunnen zien wat de gevolgen daarvan zijn. Met andere woorden: welke taken worden dan ook verplaatst?
\end{itemize}
\subsection{Opdrachtformulering}
\subsection{Op te leveren producten en diensten}
\subsection{Eisen en beperkingen}
\subsection{Voorwaarden}
