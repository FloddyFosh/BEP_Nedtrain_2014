\section{Aanpak en Tijdsplanning}
De aanpak van het hele project zal in dit hoofdstuk beschreven worden samen met een tijdsplanning. Op deze manier wordt er duidelijkheid gecre\"eerd over hoe wij de eerder gestelde eisen aan zullen pakken en in welke fasering van het project.

\subsection{Aanpak}
Het hele project zal opgedeeld worden in drie fasen, namelijk een ori\"entatie- en ontwerp-fase, een implementatie- en testen-fase en uiteindelijk een afrondingsfase. Voor elk van deze fasen zullen we bespreken wat er gedaan gaat worden, welke methode we gebruiken en waarom we hiervoor gekozen hebben. 

\subsubsection{Ori\"entatie en Ontwerp}
In de eerste twee weken houden wij ons bezig met de ori\"entatie en ontwerp van de opdracht. Tijdens de ori\"entatie moeten we bepalen wat de exacte opdracht is, wie er betrokken zijn, welke middelen we tot onze beschikking hebben en onder welke condities de opdracht gedaan moet worden. Dit soort informatie kunnen wij verkrijgen door een gesprek te voeren met de opdrachtgever en hun aangeleverde documenten en informatie te lezen. Dit alles zal resulteren in een \emph{Ori\"entatieverslag}.

Als het fundamentele gedeelte van de ori\"entatie-fase bekend is kan de focus gelegd worden op het ontwerp en de omvang van het te maken systeem. Dit kan gedaan worden door een techniek \emph{requirement analysis} toe te passen. Hierbij wordt onderzocht aan welke eisen het systeem moet voldoen en welke functionaliteit aanwezig moeten zijn. Hierbij kan onderscheid gemaakt worden tussen functionele en non-functionele requirements. De functionele requirements kunnen gevonden worden door scenario's te schrijven. De non-functionele requirements moeten voornamelijk gecommuniceerd worden door de opdrachtgever.

\subsubsection{Implementatie en Testen}
Aan de hand van de eisen en ontwerpen die beslist zijn in de ori\"entatie- en ontwerp-fase wordt het systeem ge\"implementeerd. Deze fase duurt zes weken en hierin wordt de techniek Scrum gebruikt, beschreven in \emph{The Scrum Guide} \ref{schwaber2011}. Scrum is een framework voor het ontwikkelen van software die ervoor zorgt dat de opdrachtgever en andere belanghebbenden betrokken zijn bij de implementatie, door hun de gemaakte voortgang en tot dan gemaakte software te laten zien. Hierdoor kan er ook betere terugkoppeling gegeven worden aan de projectleden, wat voor een betere samenwerking zorgt en zorgt voor een betere vervulling van de wensen van de opdrachtgever. Elke week is er een sprint waarbij gekozen wordt wat er die week ge\"implementeerd gaat worden. Aan het einde van de week zal de gemaakte functionaliteit voorgelegd worden aan de opdrachtgever, zodat de feedback gebruikt kan worden voor de volgende sprint. Dagelijks zal er ook een scrummeeting zijn van ongeveer een kwartier.

In deze fase zal de gemaakte code ook een keer opgestuurd moeten worden naar de Software Improvement Group (SIG). SIG beoordeeld de kwaliteit van de code handmatig en met behulp van geautomatiseerde tools. Het opsturen van de code naar SIG moet gebeuren op 13 Juni 2014. Ongeveer een week later ontvangen wij de feedback van hun.

\subsubsection{Afronding}
In de laatste fase van het project wordt het afronden gedaan. De feedback die gekregen is van SIG zal hier toegepast moeten worden in het systeem. Daarnaast moet de code zo afgeleverd worden dat het herbruikbaar is voor toekomstige projecten. De definitieve code moet opnieuw uiterlijk 5 dagen voor de datum van de eindpresentatie ingeleverd worden. Het maken van het eindverslag en het voorbereiden van de eindpresentatie zal ook in deze fase gedaan worden.

\subsection{Tijdsplanning}
Een tijdsplanning voor de beschreven aanpak kan gevonden worden in Bijlage \ref{app:A}.\\

