\section{Aanpak en Tijdsplanning}
De aanpak van het hele project zal in dit hoofdstuk beschreven worden samen met een tijdsplanning. Op deze manier wordt er duidelijkheid gecre\"eerd over hoe wij de eerder gestelde eisen aan zullen pakken en in welke fasering van het project.

\subsection{Aanpak}
Het hele project zal opgedeeld worden in drie fasen, namelijk een ori\"entatie- en ontwerpfase, een implementatie- en testfase en uiteindelijk een afrondingsfase. Voor elk van deze fasen zullen we bespreken wat er gedaan gaat worden, welke methode we gebruiken en waarom we hiervoor gekozen hebben. 

\subsubsection{Ori\"entatie en Ontwerp}
In de eerste twee weken houden wij ons bezig met de ori\"entatie en het ontwerp van de opdracht. Tijdens de ori\"entatie moeten we bepalen wat de exacte opdracht is, wie er betrokken zijn, welke middelen we tot onze beschikking hebben en onder welke condities de opdracht gedaan moet worden. Dit soort informatie kunnen wij verkrijgen door een gesprek te voeren met de opdrachtgever en de aangeleverde documenten en informatie te lezen. Dit alles zal resulteren in een \emph{Ori\"entatieverslag}.

Als het fundamentele gedeelte van de ori\"entatiefase bekend is, kan de focus gelegd worden op het ontwerp en de omvang van het te maken systeem. Dit kan gedaan worden door een \emph{requirement analysis} toe te passen. Hierbij wordt onderzocht aan welke eisen het systeem moet voldoen en welke functionaliteiten aanwezig moeten zijn. Bovendien kan onderscheid gemaakt worden tussen functionele en non-functionele requirements. De functionele requirements kunnen gevonden worden door scenario's te schrijven. De non-functionele requirements moeten voornamelijk gecommuniceerd worden door de opdrachtgever. Dit zal beschreven worden in het document \emph{Requirementsanalyse}.

\subsubsection{Implementatie en Testen}
\label{subsubsec:testen}
Aan de hand van de eisen en ontwerpen die beslist zijn in de ori\"entatie- en ontwerpfase wordt het systeem ge\"implementeerd. Deze fase duurt zes weken en hierin wordt de techniek Scrum gebruikt, beschreven in \emph{The Scrum Guide} \cite{schwaber2011}. Scrum is een framework voor het ontwikkelen van software dat ervoor zorgt dat de opdrachtgever en andere belanghebbenden betrokken zijn bij de implementatie, de gemaakte voortgang en de tot dan toe gemaakte software. Hierdoor kan er ook betere terugkoppeling gegeven worden aan de projectleden, wat zorgt voor een betere samenwerking en een betere vervulling van de wensen van de opdrachtgever. Het ontwikkelingsproces bestaat uit meerdere sprints, die in dit geval elk een week duren. Van tevoren is er besloten op welke taak gefocust zal worden in welke week. Aan het begin van elke sprint wordt er besproken hoe het met de planning staat, wat er gedaan is en wat er in die sprint ge\"implementeerd gaat worden. Deze besluiten worden in elke sprint nog dagelijks bekeken door het houden van scrummeetings van ongeveer een kwartier. Aan het einde van de sprint zal de gemaakte functionaliteit voorgelegd worden aan de opdrachtgever, zodat de feedback gebruikt kan worden voor de volgende sprint.

In deze fase zal de gemaakte code ook een keer opgestuurd moeten worden naar de \emph{Software Improvement Group (SIG)}. SIG beoordeelt de kwaliteit van de code handmatig en met behulp van geautomatiseerde tools. Het opsturen van de code naar SIG moet gebeuren op 13 juni 2014. Ongeveer een week later ontvangen wij hun feedback.

\subsubsection{Afronding}
De laatste fase zal bestaan uit het afronden van het project. De feedback die gekregen is van SIG zal dan toegepast moeten worden op de code. Daarnaast moet de code zo afgeleverd worden dat het herbruikbaar is voor toekomstige projecten. De definitieve code moet opnieuw, uiterlijk 5 werkdagen voor de datum van de eindpresentatie, ingeleverd worden. Het afronden van het eindverslag en het voorbereiden van de eindpresentatie zal ook in deze fase gedaan worden.

\subsection{Tijdsplanning}
Een tijdsplanning voor de beschreven aanpak kan gevonden worden in Bijlage \ref{app:A}.
