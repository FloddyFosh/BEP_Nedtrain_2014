\section{Projectinrichting}
Het doel van de projectinrichting is om inzicht te geven in de manier waarop het project georganiseerd is en hoe er voor gezorgd wordt dat het project op een gestructureerde manier verloopt.

\subsection{Organisatie}
Hoewel alle projectleden aan elk onderdeel van het project bijdragen, definiëren we zes onderdelen met voor elk onderdeel een projectlid dat daarvoor eindverantwoordelijk is. Dit projectlid is verantwoordelijk voor de kwaliteit van dit onderdeel en controleert of aan alle voorwaarden is voldaan. Ook herinnert hij de andere projectleden aan deadlines van dit onderdeel en spoort hij ze aan om hieraan te werken als dit nodig is. De verdeling van verantwoordelijkheden is weergegeven in Tabel \ref{tbl:organisatie}.

\begin{table}[!h]
\label{tbl:organisatie}
\def\arraystretch{1.5}
\begin{tabularx}{\textwidth}{|p{2cm}|X|}
\hline
\textbf{Projectlid} & \textbf{Verantwoordelijkheid} \\ \hline
Anton & Het porten van de applicatie naar Windows 7 en het onderzoeken van de mogelijkheden op dit gebied. \\ \cline{2-2}
& Het controleren of de verslagen aan de gestelde eisen voldoen en kwaliteitsbewaking over bijvoorbeeld spelling en stijl van de verslagen. \\ \hline
Chris & Ervoor zorgen dat er voldoende en kwalitatief goede tests geschreven worden, maar ook dat falende tests zo snel mogelijk opgelost worden, zodat bugs niet opstapelen. \\ \cline{2-2}
& Het zo veel mogelijk naleven van de tijdsplanning, zoals het ervoor zorgen dat er niet te lang op één probleem wordt gefocust terwijl andere taken verwaarloosd worden. \\ \hline
Martijn & Het controleren van de voortgang en de kwaliteit van de solver die gebruikt maakt van chaining. \\ \cline{2-2}
& Het controleren van de voortgang en de kwaliteit van de GUI en de veranderingen die hieraan ge\"implementeerd zullen worden. \\ \hline
\end{tabularx}
\caption{De verdeling van verantwoordelijkheden onder de projectleden.}
\end{table}

\subsection{Tijdsinvestering en vaardigheden}
De projectleden zijn in dienst als stagiair bij NedTrain en worden volgens de arbeidsovereenkomst verwacht om gemiddeld minimaal 36 uur per week aan het werk te zijn. Het Bachelor Eindproject is een studieonderdeel van 15 ECTS en heeft dus bij een studielast van 28 uur per ECTS een totale studielast van 420 uur. Over een periode van 10 weken is de verwachte studielast dus gemiddeld 42 uur per week.

Als toelatingseis voor het Bachelor Eindproject moeten de projectleden alle vakken uit het eerste en het tweede jaar van de bachelor Technische Informatica gehaald hebben. De vaardigheden die van de projectleden verwacht worden, bestaan dus uit alle vaardigheden die worden verwacht bij deze vakken. Deze zijn te vinden in de studiegids van de TU Delft \footnote{\href{http://studiegids.tudelft.nl}{studiegids.tudelft.nl}}.

\subsection{Administratieve Procedures}
Voor het bijhouden van bugs en taken zal gebruikt gemaakt worden van de ingebouwde issue tracker op Bitbucket \footnote{\href{http://bitbucket.org}{bitbucket.org}}. Aangezien de repository ook gehost is op Bitbucket, is het makkelijk om administratie hier ook bij te houden. Met deze tracker kan van een issue geselecteerd worden of het een bug, enhancement, proposal of een task is. Daarnaast kan een prioriteit ingesteld worden en kan de taak toegewezen worden aan een projectlid.

Voor een aantal uitgebreidere trackers, zoals Atlassian Jira \footnote{\href{http://atlassian.com/software/jira}{atlassian.com/software/jira}} of Mantis Bug Tracker \footnote{\href{http://mantisbt.org}{mantisbt.org}} moet betaald worden. Wij verwachten dat de Bitbucket tracker ook voldoende mogelijkheden biedt.

\subsection{Rapportering}
Het team zal communiceren met de opdrachtgever op verschillende manieren. E\'en daarvan is het in een gesprek doornemen wat voor code er is gemaakt en wat er in de verslagen opgeschreven is. Er worden vier verslagen gemaakt: het Plan van Aanpak, het Ori\"entatieverslag, de Requirementsanalyse en het Eindverslag. Bij de gesprekken maken we notulen, zodat we de afspraken en idee\"en die ter sprake komen niet vergeten.

\subsection{Financiering en Resources}
\label{subsec:resources}
Voor dit project wordt er zoveel mogelijk gebruik gemaakt van open source software. We verwachten dat we alle doelen kunnen behalen zonder aparte licenties nodig te hebben.

Het team zorgt zelf voor computers waar zowel Windows 7 als een versie van Ubuntu op is ge\"installeerd. De tool die wij gaan verbeteren is gemaakt in \cpp\ met behulp van het framework Qt\footnote{\href{http://qt-project.org}{qt-project.org}}, dat het mogelijk maakt om eenvoudig cross-platform GUI's te ontwikkelen voor \cpp\ software. Door een aantal grote wijzigingen in Qt werkt de tool nog wel in Qt versie 4.8, maar niet met Qt versie 5.x. We maken ook gebruik van een private repository op Bitbucket als versiebeheersysteem in combinatie met de ingebouwde issue tracker. Tenslotte wordt er gebruik gemaakt van de continuous integration service Jenkins, dat wordt gehost op een eigen server op eigen apparatuur om zo kosten voor het huren van een server te besparen.

Op de TU Delft zijn er genoeg werkplekken, maar bij NedTrain is er niet veel ruimte om te werken aan het project. Het maakt voor het team weinig uit waar er gewerkt wordt, thuis werken zou ook een mogelijkheid kunnen zijn. We kiezen er toch voor om zoveel mogelijk op dezelfde locatie te werken, omdat we verwachten dat dit de samenwerking en het product ten goede zal komen. 
