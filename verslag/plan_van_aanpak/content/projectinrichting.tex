\section{Projectinrichting}
\subsection{Organisatie}
\subsection{Personeel}
\subsection{Administratieve procedures}
\subsection{Financiering}
\subsection{Rapportering}
Het team gaat communiceren met de opdrachtgever op verschillende manieren. E\'en daarvan is in een gesprek doornemen wat we voorlopig hebben gemaakt en hebben opgeschreven in het \'e\'en van de verslagen. Er worden drie verslagen gemaakt: het Plan van Aanpak, het ori\"entatieverslag en het eindverslag. Bij de gesprekken maken we notulen, zodat we de afspraken en idee\"en die ter sprake komen niet vergeten.

\subsection{Resources}
Voor dit project wordt er zoveel mogelijk gebruik gemaakt van open source software. We verwachten dat we alles doelen kunnen behalen zonder aparte licenties nodig te hebben. \\

Het team zorgt zelf voor computers waar zowel Windows 7 als een versie van Ubuntu op is ge\"installeerd. De tool die wij gaan verbeteren is gemaakt in C++ met behulp van het framework Qt\footnote{Meer informatie is te vinden op qt-project.org}, dat het mogelijk maakt om eenvoudig cross-platform GUI's te ontwikkelen voor C++ software. Ook wordt er gebruikt gemaakt van een private repository op bitbucket.com als versie beheersysteem. \\

Op de TU Delft zijn er genoeg werkplekken, maar bij NedTrain is er niet veel ruimte om te werken aan het project. Het maakt voor het team weinig uit waar we werken, dit zou ook zelfs thuis kunnen. We kiezen er toch voor om zoveel mogelijk op dezelfde locatie te werken, omdat we verwachten dat dit de samenwerking en het product ten goede zal komen. 
