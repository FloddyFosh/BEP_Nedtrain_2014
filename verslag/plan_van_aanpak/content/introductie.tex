\section{Inleiding}
De introductie is gericht op het Plan van Aanpak en het tot stand komen ervan. Hierin wordt onder andere beschreven wat de aanleiding is van de projectopdracht en dit Plan van Aanpak. Daarnaast zal hierin vastgesteld worden op welke wijze het Plan van Aanpak wordt goedgekeurd en bijgesteld.\\

\subsection{Aanleiding}
Voor een uitbreiding en verbetering van een softwaretool was het bedrijf NedTrain op zoek naar een groep studenten. Aan ons werd gevraagd of wij belangstelling hadden om hieraan te werken voor ons Bachelorproject. Op deze manier zijn wij indirect in contact gekomen met NedTrain. Tijdens een eerste kennismakingsgesprek zijn de opdracht en belangen van NedTrain besproken. Naar aanleiding van dit gesprek is dit Plan van Aanpak opgesteld.\\

\subsection{Opdrachtgever}
Zoals eerder vermeld, is NedTrain samen met de TU Delft onze opdrachtgever. Het contactpersoon binnen NedTrain is ir. Bob Huisman (b.huisman@nedtrain.nl). Hij heeft ook de vorige projecten die zich bezig hielden met de 'NedTrain Planner' begeleid. Onze begeleider van de TU Delft is prof. dr. Cees Witteveen (c.witteveen@tudelft.nl). Hij is werkzaam binnen de Algoritmiek Groep op de faculteit EWI (Elektrotechniek, Wiskunde en Informatica).\\

\subsection{Accordering en bijstelling}
Dit Plan van Aanpak zal worden besproken en gekeurd door de twee eerder genoemde begeleiders Bob Huisman en Cees Witteveen. Ook als we iets willen aanpassen aan dit Plan van Aanpak, dan zal dat eerst goedgekeurd moeten worden door de begeleiders.\\

\subsection{Toelichting op de opbouw van het plan}
