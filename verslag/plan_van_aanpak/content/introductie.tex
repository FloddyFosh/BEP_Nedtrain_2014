\section{Inleiding}
De introductie is gericht op het Plan van Aanpak en het tot stand komen ervan. Hierin wordt onder andere beschreven wat de aanleiding is van de projectopdracht en dit Plan van Aanpak. Daarnaast zal hierin vastgesteld worden op welke wijze het Plan van Aanpak wordt goedgekeurd en bijgesteld.\\

\subsection{Aanleiding}
Voor een uitbreiding en verbetering van een softwaretool was het bedrijf NedTrain op zoek naar een groep studenten. Aan ons werd gevraagd of wij belangstelling hadden om hieraan te werken voor ons Bachelorproject. Op deze manier zijn wij indirect in contact gekomen met NedTrain. Tijdens een eerste kennismakingsgesprek zijn de opdracht en belangen van NedTrain besproken. Naar aanleiding van dit gesprek is dit Plan van Aanpak opgesteld.\\

\subsection{Accordering en bijstelling}
Het Plan van Aanpak zal worden goedgekeurd door dit te versturen naar NedTrain en onze begeleiders. Eventuele nodige bijstellingen zullen gecommuniceerd en besproken worden en aangepast als er door beide kanten goedkeuring gegeven is. Latere aanpassingen aan het Plan van Aanpak die nodig zijn zullen in de voortgangsrapportages opgenomen worden.\\

\subsection{Toelichting op de opbouw van het plan}
In de volgende hoofdstuk zal de precieze opdracht beschreven worden. Informatie over de opdrachtgever en begeleiders zullen gegeven worden en de gestelde eisen en randvoorwaarden worden vastgesteld. In de hoofdstukken daarna zal een antwoord gegeven worden op de vraag hoe er te werk zal worden gegaan. Een tijdsplanning zal gegeven worden en een beschrijving van de projectinrichting. Afsluitend zal een inzicht gegeven worden in de relatie tussen de voorgestelde maatregelen en de door de opdrachtgever gestelde eisen ten aanzien van de kwaliteit.\\