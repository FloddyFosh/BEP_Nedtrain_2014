\section{Inleiding}
\subsection{Aanleiding}
We zijn in contact gekomen met NedTrain, die software gebruikt om een planning te maken wanneer welke trein gerepareerd wordt. Deze software is gemaakt door studenten van de TU Delft. NedTrain wil graag een aantal functies toevoegen aan deze software. Wat wij gaan doen voor NedTrain leggen wij vast in dit document. 

\subsection{Opdrachtgever}
Zoals eerder vermeld, is NedTrain samen met de TU Delft onze opdrachtgever. Het contactpersoon binnen NedTrain is ir. Bob Huisman (b.huisman@nedtrain.nl). Hij heeft ook de vorige projecten die zich bezig hielden met de 'NedTrain Planner' begeleid. Onze begeleider van de TU Delft is prof. dr. Cees Witteveen (c.witteveen@tudelft.nl). Hij is werkzaam binnen de Algoritmiek Groep op de faculteit EWI (Elektrotechniek, Wiskunde en Informatica). 

\subsection{Accordering en bijstelling}
Dit Plan van Aanpak zal worden besproken en gekeurd door de twee eerder genoemde begeleiders Bob Huisman en Cees Witteveen. Ook als we iets willen aanpassen aan dit Plan van Aanpak, dan zal dat eerst goedgekeurd moeten worden door de begeleiders.

\subsection{Toelichting op de opbouw van het plan}
