\section{Kwaliteitsborging}
\subsection{Kwaliteitsbewaking}
\subsection{Versiebeheer}
Er zal gebruikt gemaakt worden van een private repository op bitbucket.com dat werkt met Git. Subversion behoorde ook tot de mogelijkheden, maar omdat het team beter bekend is met git en dit ook populairder is onder andere ontwikkelaars hebben we de voordelen van Subversion niet bekeken. Subversion wordt gratis beschikbaar gesteld voor studenten door de faculteit EWI. Daarentegen kan Git gratis gebruikt worden door iedereen en bestaat ook zo de mogelijkheid om later het project publiekelijk toegankelijk te maken. Er is voldoende documentatie te vinden voor Git als we ergens niet uitkomen. Er is gekozen voor de website bitbucket.com gekozen en niet voor het bekendere github.com, omdat op bitbucket gemakkelijk en gratis een repository aangemaakt kan worden die alleen door het team bekeken kan worden. 
