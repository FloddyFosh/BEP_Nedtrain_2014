\section{Kwaliteitsborging}
In dit hoofdstuk zal uitgewerkt worden welke middelen gebruikt zullen worden om bij te dragen aan de kwaliteit van het product.

\subsection{Kwaliteitsbewaking}
Om de werking van de code te garanderen, zal uiteraard testcode geschreven worden. Deze code zal bovendien gedraaid worden door een continuous integration service, bijvoorbeeld Jenkins \footnote{\url{jenkins-ci.org}}, zodat er gelijk een melding wordt gedaan van een testcase die niet slaagt.

Zoals eerder genoemd in sectie \ref{subsec:resources}, wordt er gebruik gemaakt van de issue tracker van Bitbucket. Hierop zullen gevonden bugs geplaatst worden en zal bijgehouden worden welke features al ge\"implementeerd zijn en welke niet.

\subsection{Versiebeheer}
Er zal gebruikt gemaakt worden van een private repository op Bitbucket dat werkt met Git. Subversion behoorde ook tot de mogelijkheden, maar omdat het team beter bekend is met git en dit ook populairder is onder andere ontwikkelaars hebben we voor een Git repository gekozen. Subversion wordt gratis beschikbaar gesteld voor studenten door de faculteit EWI. Daarentegen kan Git gratis gebruikt worden door iedereen en bestaat ook zo de mogelijkheid om later het project publiekelijk toegankelijk te maken. Er is voldoende documentatie te vinden voor Git als we ergens niet uitkomen. Er is voor de website Bitbucket gekozen en niet voor het bekendere Github.com, omdat op Bitbucket gemakkelijk en gratis een repository aangemaakt kan worden die alleen door het team bekeken kan worden. 
