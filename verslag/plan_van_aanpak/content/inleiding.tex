\section{Inleiding}
De inleiding is gericht op het Plan van Aanpak en het tot stand komen ervan. Hierin wordt onder andere beschreven wat de aanleiding is van de projectopdracht en dit Plan van Aanpak. Daarnaast zal hierin vastgesteld worden op welke wijze het Plan van Aanpak zal worden goedgekeurd en bijgesteld.

\subsection{Aanleiding}
Voor een uitbreiding en verbetering van een softwaretool was het bedrijf NedTrain op zoek naar een groep studenten. Aan ons werd gevraagd of wij belangstelling hadden om hieraan te werken voor ons Bachelorproject. Dit leek ons interessant om te doen en dus zijn wij indirect in contact gekomen met NedTrain. Tijdens een eerste kennismakingsgesprek zijn de opdracht en belangen van NedTrain besproken. Naar aanleiding van dat gesprek is dit Plan van Aanpak opgesteld.

\subsection{Accordering en bijstelling}
Het Plan van Aanpak zal worden goedgekeurd door dit te versturen naar NedTrain en de begeleiders. Eventuele nodige bijstellingen zullen aangegeven en besproken worden en aangepast als er door beide kanten hiervoor goedkeuring gegeven is. Latere aanpassingen aan het Plan van Aanpak die nodig zijn, zullen in de voortgangsrapportages opgenomen worden.

\subsection{Toelichting op de opbouw van het plan}
In het volgende hoofdstuk zal de precieze opdracht beschreven worden. Hierin maken we informatie over de opdrachtgever en begeleiders bekend. Daarnaast we de eisen en randvoorwaarden vast. In de hoofdstukken daarna zal een antwoord gegeven worden op de vraag hoe er te werk zal worden gegaan tijdens dit project. Een tijdsplanning en een beschrijving van de projectinrichting zullen hier gegeven worden. Afsluitend zal inzicht gegeven worden over hoe de kwaliteit van het project en de code gewaardborgd blijft.