\section{Inleiding}
Dit project zal gaan over het implementeren van een nieuw algoritme voor een bestaande grafische scheduling-tool en het uitbreiden en verbeteren hiervan voor het bedrijf \emph{NedTrain}. Met deze tool kunnen gebruikers complexe schedulingsproblemen oplossen en op een intu\"itieve manier stapsgewijs onderzoeken. Om het project zo gestructureerd mogelijk te laten verlopen wordt er van tevoren een Plan van Aanpak geschreven. Een Plan van Aanpak bevat alle gemaakte afspraken tussen de opdrachtgever en de projectleden en beschrijft op welke manier het project aangepakt gaat worden. Het bevat onder andere een precieze beschrijving van de opdracht, de gestelde eisen, de op te leveren producten, beperkingen en de voorwaarden. Daarnaast geven wij een tijdsplanning en beschrijven we hoe de kwaliteit van de gemaakte code en het project gewaarborgd wordt. Dit zorgt ervoor dat het gehele project en de uitvoering daarvan helder is voor alle betrokkenen. Wijzigingen en toevoegingen aan het Plan van Aanpak kunnen eventueel gedurende het project besproken worden, zolang de opdrachtgever alsmede de projectleden hiervoor goedkeuring hebben gegeven. Deze wijzigingen zullen opgenomen worden in de voortgangsrapportages.

\subsection{Het Bedrijf}
NedTrain BV is een dochterbedrijf van de \emph{Nederlandse Spoorwegen (NS)} dat zorgt voor het onderhouden en repareren van de treinen die NS gebruikt. \cite{NedTrainSite} Bovendien zijn ze ook verantwoordelijk voor het onderhoud van het materieel van Arriva, Syntus en Veolia. NedTrain werkt 24 uur per dag, 7 dagen per week met ongeveer 3500 mensen verspreid over meer dan 30 vestigingen door heel Nederland. Het hoofdkantoor van NedTrain is gevestigd in Utrecht.

\subsection{Aanleiding Project}
Voor NedTrain is dit project onderdeel van onderzoek naar de mogelijkheden van complexe algoritmen om hun taken te automatiseren. In het verleden is er door NedTrain, in samenwerking met de TU Delft, een tool ontwikkeld waarin het plannen van taken gevisualiseerd kan worden. Deze tool, de \emph{NedTrain planner}, wordt gebruikt door NedTrain om onderzoek te doen naar schedulingsalgoritmen en het verbeteren daarvan. Voor de TU Delft heeft deze tool ook een educatief doel, namelijk het op een visuele manier duidelijk maken wat er gebeurt bij het uitvoeren van een complex algoritme.

Naar aanleiding van de laatste vooruitgang op het gebied van schedulingsalgoritmen, bijvoorbeeld door Wilson et. al. op het gebied van flexibiliteit \cite{wilson2013flexibility}, wil NedTrain graag dat hun tool verbeterd wordt om deze vooruitgang te implementeren. Aan ons werd gevraagd of wij belangstelling hadden om hieraan te werken voor ons Bachelorproject. Dit leek ons interessant om te doen en dus zijn wij indirect in contact gekomen met NedTrain. Tijdens een eerste kennismakingsgesprek zijn de opdracht en belangen van NedTrain besproken. Naar aanleiding van dat gesprek is dit Plan van Aanpak opgesteld.
