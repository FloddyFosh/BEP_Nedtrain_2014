\section{Inleiding}
Dit project zal gaan over het uitbreiden en verbeteren van een bestaande grafische scheduling-tool voor het bedrijf NedTrain. Met deze tool kunnen gebruikers complexe schedulingsproblemen oplossen en op een intu\"itieve manier stapsgewijs onderzoeken. Om het project zo gestructureerd mogelijk te laten verlopen wordt er van tevoren een Plan van Aanpak geschreven. Een Plan van Aanpak bevat alle gemaakte afspraken tussen de opdrachtgever en de projectleden en beschrijft op welke manier het project aangepakt gaat worden. Het bevat onder andere een precieze beschrijving van de opdracht, de gestelde eisen, de op te leveren producten en beperkingen. Daarnaast geven wij een tijdsplanning en beschrijven we hoe de kwaliteit van de gemaakte code en het project gewaarborgd wordt. Dit zorgt ervoor dat het gehele project en de uitvoering daarvan helder is voor alle betrokkenen. Wijzigingen en toevoegingen aan het Plan van Aanpak kunnen eventueel gedurende het project besproken worden, zolang de opdrachtgever alsmede de projectleden hiervoor goedkeuring hebben gegeven. Deze wijzigingen zullen opgenomen worden in de voortgangsrapportages.

\subsection{Het Bedrijf}
NedTrain BV is een dochterbedrijf van de Nederlandse Spoorwegen (NS) dat zorgt voor het onderhouden en repareren van de treinen die NS gebruikt. Hieronder valt ook het materieel van Arriva, Syntus en Veolia. NedTrain werkt 24 uur per dag, 7 dagen per week met ongeveer 3500 mensen verspreid over 30 vestigingen door heel Nederland. Het hoofdkantoor van NedTrain is gevestigd in Utrecht.

\subsection{Aanleiding Project}
Voor de uiteenlopende taken die NedTrain uitvoert, wordt geprobeerd zoveel mogelijk geautomatiseerde software te gebruiken. Onder andere bezit NedTrain software die complexe schedulings-problemen kan oplossen en gebruikt kan worden om werknemers op een educatieve manier inzicht te geven over de werking ervan. In het verleden is er door NedTrain, in samenwerking met de TU Delft, een tool ontwikkeld waarin het plannen van taken gevisualiseerd kan worden. Deze tool, de \emph{NedTrain planner}, wordt gebruikt door NedTrain om onderzoek te doen naar scheduling-algoritmes en het verbeteren daarvan. Daarnaast kan het ook goed gebruikt worden als basis om werknemers gevoel te laten krijgen voor het plannen van taken. Volgens NedTrain mist deze tool enige functionaliteit en zou er hier en daar wat verbeterd kunnen worden. Het bedrijf was op zoek naar een groep studenten om deze functionaliteit te implementeren en de tool te verbeteren. Aan ons werd gevraagd of wij belangstelling hadden om hieraan te werken voor ons Bachelorproject. Dit leek ons interessant om te doen en dus zijn wij indirect in contact gekomen met NedTrain. Tijdens een eerste kennismakingsgesprek zijn de opdracht en belangen van NedTrain besproken. Naar aanleiding van dat gesprek is dit Plan van Aanpak opgesteld.
