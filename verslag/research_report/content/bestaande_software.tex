\section{Bestaande Software}
Het doel van ons project is het verbeteren van bestaande software de 'NedTrain Planner'. Deze software maakt gebruik van frameworks. In dit hoofdstuk wordt uiteengezet met welke software er gebruikt gaat worden. Eerst bespreken we de 'NedTrain Planner' zelf en daarna bespreken we het framework Qt, dat is gebruikt voor de tool. 

\subsection{NedTrain Planner}
De 'NedTrain Planner' bestaat grofweg uit drie componenten namelijk de gebruikers interface, een solver en de communicatie tussen die twee. De gebruikers interface maakt het mogelijk om een probleeminstantie te maken, deze op te lossen door de solver en de oplossing te bekijken. De solver lost het probleem op. \\

In vorige bachelorprojecten en masterthesissen is deze tool steeds verder uitgebreid. De 'OnTrack Scheduler' is een tool met een eenvoudige GUI en een solver. Zijn onderzoek in 2010 voor de masterthesis had als doel om een goede solver te bouwen. \cite{ronaldevers2010} \\

Daarna in 2011 door Edwin Bruurs en Cis van der Louw een verbeterde gebruikers interface gemaakt.\cite{bep2011nedtrain} Deze verbeteringen waren gebruikers vriendelijk, maar de code kwaliteit was niet goed volgens het bachelorproject uit 2012.\cite{bep2012nedtrain} In 2012 hebben Erik Ammeraan, Jan Elffers, Erwin Walraven en Wilco Wisse opnieuw een verbeterde gebruikers interface gemaakt voor de 'OnTrack Scheduler' deze hebben zij de 'NedTrain Planner' genoemd.

\subsection{Qt Framework}
Qt is een gebruikers interface framework voor \cpp . De 'NedTrain Planner' maakt gebruik van versie 4.8. De huidige nieuwste versie is 5.2. Er zijn een aantal verschillen tussen 4.8 en 5.2, sinds 5.0 waardoor de 'NedTrain Planner' niet werkt op 5.2. Qt is een open source software dat voor meerdere besturingsystemen geschikt is, het moet dus mogelijk zijn om de tool beschikbaar te maken voor o.a.: Windows, Linux en Mac. 
