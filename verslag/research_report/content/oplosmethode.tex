\section{Oplossingsmethode}
In dit hoofdstuk wordt besproken hoe we het probleem gaan aanpakken. 

\subsection{Oplossing}
Chaining

Voor het oplossen van Linear Programming problemen bestaan er al verschillende solvers. E\'en daarvan is Gurobi commercieel software pakket, dat de problemen zeer snel kan oplossen. COIN-OR LP\footnote{projects.coin-or.org/Clp}, afgekort Clp, is een open source Linear Programming solver in \cpp . We kiezen voor Clp, omdat deze software open source is, iets wat voor onze opdrachtgever aantrekkelijker is dan Gurobi, omdat daar wel een licentie voor gekocht moet worden als de 'NedTrain Planner' echt gebruikt gaat worden.

\subsection{Werkwijze}
Tijdens de ontwikkelfase gaat het team werken met scrum in wekelijkse sprints. 

\subsection{Hulpmiddelen}
Als versiebeheersysteem gebruiken we BitBucket. Op deze website zijn private Git repositories gratis beschikbaar. Bij de concurrent GitHub zijn ook private Git repositories beschikbaar, maar dan moet het account wel geactiveerd worden met een TU Delft e-mailadres. Op BitBucket is dit dus eenvoudiger in te stellen. We hebben voor Git gekozen en niet voor Subversion, omdat we meer ervaring hebben met Git. BitBucket heeft ook een ingebouwd issue-trackingsysteem waarvan wij gebruik gaan maken. Dit is in Subversion minder makkelijk in te stellen.

Om ervoor te zorgen dat we altijd werkende code hebben op Bitbucket, gaan we continuous integration gebruiken. Daarvoor is een server waarop Jenkins is ge\"intalleerd erg geschikt. Stel dat we code uploaden die niet werkt, dan geeft Jenkins daarvan automatisch een melding, zodat wij de code kunnen verbeteren en we weer werkende code hebben op Bitbucket.  

Om er voor te zorgen dat onze code goed onderhoudbaar blijft, gaan we ook een stylechecker gebruiken. De stylechecker \emph{cppcheck} controleert statisch de code op fouten en geeft waarschuwingen wanneer er bijvoorbeeld variabelen worden gedeclareerd die nooit gebruikt worden. De stylechecker \emph{cpplint} controleert op de Google code guideline voor \cpp. Ook \emph{cpplint} is niet perfect, zo zeggen ze zelf, omdat het niet alle fouten ontdekt, maar wij denken dat dit wel een goede toevoeging is. We hebben ook naar de alternatieven \emph{Vera\texttt{++}} en \emph{cxxchecker}, maar deze waren niet gemakkelijk te installeren.
