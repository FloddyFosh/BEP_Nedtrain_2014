\section{Bestaande Software}
Onze opdracht is het verbeteren en uitbreiden van de 'NedTrain planner', die het resultaat is van meerdere projecten door studenten van de TU Delft. In dit hoofdstuk wordt uiteengezet welke software er al bestaat en wat voor software gebruikt gaat worden gedurende het project. Eerst bespreken we de NedTrain planner zelf, daarna het framework Qt dat gebruikt is voor het implementeren van de tool en tenslotte de Google Test library en de solvers.

\subsection{NedTrain Planner}
\label{subsec:planner}
De NedTrain Planner bestaat grofweg uit drie componenten, namelijk de gebruikers-interface, een solver en de communicatie tussen die twee. De gebruikers-interface maakt het mogelijk om een probleeminstantie, dat beschreven zal worden in \ref{subsec:probleem}, te maken en oplossingen hiervan door een solver te visualiseren.

In vorige bachelorprojecten en mastertheses is deze tool steeds verder uitgebreid. De 'OnTrack Scheduler' is een tool met een eenvoudige GUI en een solver die ontwikkeld is door Ronald Evers. Zijn onderzoek in 2010 voor de masterthesis had als doel om een goede solver te bouwen \cite{ronaldevers2010}.

Daarna is in 2011 door Edwin Bruurs en Cis van der Louw een verbeterde gebruikers interface gemaakt \cite{bep2011nedtrain}. Deze verbeteringen waren gebruiksvriendelijker, maar de kwaliteit van de code was volgens het bachelorproject uit 2012 niet goed \cite{bep2012nedtrain}. Deze groep, bestaande uit Erik Ammerlaan, Jan Elffers, Erwin Walraven en Wilco Wisse, heeft in 2012 opnieuw een verbeterde gebruikers-interface gemaakt voor de 'OnTrack Scheduler'. Deze hebben zij de NedTrain planner genoemd.

In deze nieuwe versie is onder andere de gebruikersinterface uitgebreid, kan het oplosproces stap voor stap weergegeven worden en kunnen probleeminstanties gemakkelijk geladen en opgeslagen worden. De gebruikersinterface van de NedTrain planner gemaakt in het Bachelor Eindproject van 2012 is te zien in Figuur \ref{fig:plannergui}.

\begin{figure}[!h]
\label{fig:plannergui}
\centering
\includegraphics[width=\textwidth]{../images/plannergui.png}
\caption{Gebruikersinterface NedTrain planner (Bachelor Eindproject 2012)}
\end{figure}

\subsection{Qt Framework}
Qt is een gebruikers-interface framework voor \cpp. De NedTrain planner maakt op dit moment gebruik van versie 4.8, maar de nieuwste versie is nu al versie 5.2. Door een paar belangrijke veranderingen ge\"introduceerd in versie 5.0, werkt de NedTrain planner niet meer op versie 5.2. Het updaten van de NedTrain planner naar deze nieuwe versie zou mogelijk moeten zijn. Qt is open source software die voor meerdere besturingssystemen geschikt is. Het moet dus ook mogelijk zijn om de tool beschikbaar te maken voor onder andere Windows, Linux en Mac OS. Er is een IDE genaamd Qt Creator, die het mogelijk maakt om eenvoudig GUI's te ontwikkelen en \cpp\ code te compileren.

\subsection{Google Test en QTest}
Voor de bestaande software, gemaakt in \cpp, is er ook \cpp\ testcode geschreven, waarbij gebruik is gemaakt van de Google Test library. \footnote{\href{https://code.google.com/p/googletest/}{code.google.com/p/googletest/}} Dit maakt het mogelijk om unit testing te doen. Ook ondersteunt de Google Test library fixture, mocks, exceptions, macro's en templates. Er wordt in deze tests gebruik gemaakt van de QTest namespace, waar alle functionaliteit van de QTestLib in zit. De andere test-libraries die we hebben bekeken zijn: QTestLib, CppTest en CppUnit. Deze drie en de eerder genoemde Google Test library worden allemaal, met behulp van een plugin, ondersteund door Jenkins. Dat is belangrijk voor ons, omdat er dan op een relatief eenvoudige manier continuous integration toegepast kan worden. 

QTestLib is bedoeld voor het testen van Qt applicaties en bevat zowel ondersteuning voor unit testing als het testen van GUI's. Dit zou ook een goed alternatief zijn, omdat het echt bedoeld is voor het Qt framework. CppTest en CppUnit verschillen op een aantal punten. Het schijnt dat CppTest in het gebruik wat makkelijker is, maar met CppUnit is er meer mogelijk. Google Test library staat mocks toe, iets wat met CppTest en CppUnit niet mogelijk is. QTestLib staat het ook toe om het toetsenbord en de muis te simuleren. Aangezien de bestaande tests gebruik maken van de Google Test library en de QTest namespace, zijn deze allebei nodig om de bestaande tests te runnen. Daarom zal in ons project op dezelfde manier gebruik gemaakt worden van de Google Test library en de QTest functionaliteit. Dit betekent dat primair de Google Test library gebruikt zal worden en daarnaast de QTest namespace voor de GUI-tests. 

\subsection{Solver}
\label{subsec:solver}
Een programma dat een instantie van een schedulingsprobleem oplost wordt een solver genoemd. De solver die wij tot onze beschikking hebben, genaamd TMS, is ontwikkeld door \mbox{Ronald Evers} \cite{ronaldevers2010} en later verder ontwikkeld door \mbox{Michel Wilson} en de in paragraaf \ref{subsec:planner} genoemde groep voor hun Bachelor Eindproject in 2012. Een solver krijgt als input een instantie van een schedulingsprobleem bestaande uit een verzameling taken, resources en voorrangsrelaties in combinatie met enkele restricties, zoals release- en deadlinetijden en de maximale capaciteit van resources. De parser \emph{Bison}\footnote{\url{http://www.gnu.org/software/bison/}} zorgt ervoor dat de data van een dergelijke instantie omgezet wordt in een datastructuur die een solver kan begrijpen. Deze solver maakt gebruik van \emph{Precedence Constraint Posting} \cite{seminarium2014}, waarbij er op basis van een heuristiek voorrangsrelaties aan de taken worden toegevoegd om resource conflicten te verhelpen.

Tijdens het oplossen van een instantie door een solver is er continu communicatie tussen de solver en de visualisatie tool. Zo wordt er onder andere doorgegeven hoe ver de solver is met het oplossen en of er een oplossing bestaat. Als de solver een oplossing gevonden heeft geeft de solver een signaal en wordt de output ingelezen door de visualisatie tool. 

Volgens de output die gegeven is door een solver wordt de oplossing van een probleem gevisualiseerd. Zo zorgt de solver die wij tot onze beschikking hebben er voor dat er geen enkel resource conflict meer is. De methode die wordt gebruikt, garandeert echter alleen dat de \emph{Earliest Starting Time} oplossing consistent is. Dit betekent dat elke taak op de vroegst mogelijke starttijd begint. In de tool kan er ook met taken geschoven worden, maar dit kan er toe leiden dat er een resource conflict ontstaat.