\section{Inleiding}
Dit project betreft het implementeren van het chaining algoritme in combinatie met een LP solver voor een bestaande grafische scheduling-tool van het bedrijf NedTrain. Deze oplossing moet op een intu\"itieve manier gevisualiseerd worden door het uitbreiden en verbeteren van de bestaande software. Hierdoor kunnen gebruikers de flexibiliteit van complexe schedulingsproblemen oplossen en onderzoeken. In het ori\"entatieverslag wordt het vooronderzoek in de ori\"entatiefase van het project beschreven. Hierbij wordt er niet gekeken naar wat de opdrachtomschrijving is en welke eisen er gesteld worden door de opdrachtgever, aangezien dit beschreven is in het Plan van Aanpak. In dit verslag wordt er gekeken naar de achtergrond van het bedrijf en de software en methoden die voorafgaand aan het project al gebruikt werden. Er zullen beschrijvingen hiervan gegeven worden in combinatie met referenties en een uitleg waarom er voor bepaalde technieken en methoden gekozen is.
