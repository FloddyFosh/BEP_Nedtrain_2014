\section{Oplossingsmethode}
In dit hoofdstuk zal eerst het achterliggende schedulingsprobleem van de NedTrain planner gedefini\"eerd worden. 
Vervolgens komt aan bod wat voor methoden er gebruikt gaan worden om de benodigde features te implementeren, dan wat voor werkwijze er gebruikt zal worden en tenslotte wat voor hulpmiddelen hierbij nodig zijn.

\subsection{Probleem}
\label{subsec:probleem}
Het probleem dat de NedTrain planner probeert op te lossen, bestaat uit een verzameling taken, een verzameling voorrangsrelaties, een verzameling van verschillende soorten resources en voor elke resource op elk tijdstip een maximale capaciteit. Om dit probleem op te lossen, moet er voor elke taak een starttijd gevonden worden, zodat er aan alle voorrangsrelaties en alle resource-capaciteiten wordt voldaan.
Dit probleem wordt ook wel gedefini\"eerd als het \emph{Resource-constrained Scheduling Problem (RCPSP)}\cite{seminarium2014}. Dit probleem is NP-hard, dus er is geen bekend polynomiaal algoritme om dit probleem om te lossen \cite{blazewicz1983scheduling}. Omdat de probleeminstanties van NedTrain te groot zijn om exact op te lossen, moet er dus een niet-exacte methode gebruikt worden. De methode die door de huidige solver gebruikt wordt, is al beschreven in paragraaf \ref{subsec:solver}, maar deze methode zal uitgebreid worden met de methoden die beschreven zullen worden in paragraaf \ref{subsec:oplossing}.
Meer over RCPSP en de mogelijke oplossingsmethoden ervan is te vinden in het bachelorseminariumverslag dat geschreven is door de huidige projectleden en Willem-Jan Meerkerk \cite{seminarium2014}.

\subsection{Oplossing}
\label{subsec:oplossing}
De grootste hindernissen van het project zijn het implementeren van een nieuwe solver die gebruik maakt van het chaining-algoritme en flexibiliteitsintervallen berekent door middel van een linear programming solver. Het onderzoek dat naar deze problemen gedaan is en de gekozen oplossing zal in deze paragraaf besproken worden.

\subsubsection*{Chaining}
Zoals eerder genoemd in paragraaf \ref{subsec:solver}, kan een schema inconsistent worden met de resources als er in de interface taken verschoven worden. Om dit probleem op te lossen, kan de \emph{chaining} methode gebruikt worden \cite{policella2007precedence}. Voor deze methode moet eerst een oplossing voor de instantie gevonden worden die resource consistent is. Daarna wordt er aan elke eenheid van een resource een zogenaamde chain van taken toegekend. Zo'n chain is een lijst van taken $a_1, \dots a_n$, waarbij voor elke taak $a_i$ geldt dat deze taak afgerond moet zijn, voordat taak $a_{i+1}$ mag beginnen. Deze voorrangsrelatie wordt ook wel genoteerd als $a_i \prec a_{i+1}$. Pseudocode voor het chaining-algoritme is weergegeven in Algorithm \ref{alg:chaining}.

\begin{algorithm}
\caption{Chaining \cite{policella2007precedence} }\label{alg:chaining}
\textbf{Input:} A problem P and one of its fixed-times schedules $S$ \\
\textbf{Output:} A partial order solution $POS^{ch}$
\begin{algorithmic}[1]
  \Function{Chaining}{$P, S$}
    \Let{$POS^{ch}$}{$P$}
    \State Sort all the activities according to their start times in $S$  
    \State Initialize all the chains empty
    \For{\textbf{each} resource $r_j$}
      \For{\textbf{each} activity $a_i$}
        \For{$1$ \textbf{to} $req_{ij}$}
          \Let{$k$}{$SelectChain(a_i,r_j)$}
          \Let{$a_k$}{$last(k)$}
          \State{$AddConstraint(POS^{ch},a_k \prec a_i)$}
          \Let{$last(k)$}{$a_i$}
        \EndFor
      \EndFor
    \EndFor
    \State{\Return{$POS^{ch}$}}
  \EndFunction
\end{algorithmic}
\end{algorithm}

Omdat de chaining methode als input een consistent schema nodig heeft, kan de in paragraaf \ref{subsec:solver} genoemde solver gebruikt worden. Deze solver levert een earliest starting time oplossing, die vervolgens dus door het chaining algoritme gebruikt kan worden.


\subsubsection*{Linear Programming}
Voor het bepalen van de flexibiliteit van een oplossing, zullen we gebruik maken van de methode beschreven in \emph{Flexibility and Decoupling in the Simple Temporal Problem} \cite{wilson2013flexibility}. Dit paper beschrijft een metriek voor de flexibiliteit op basis van intervalsets. Om deze flexibiliteit te maximaliseren, moet het gegeven tijdschema getransformeerd worden, waarna het door middel van \emph{Linear Programming} effici\"ent opgelost kan worden. Dit geeft als resultaat voor elke taak een onafhankelijk interval waarover deze taak verschoven kan worden, zonder dat dit invloed heeft op de andere taken.

Voor het oplossen van Linear Programming problemen bestaan er al verschillende solvers. E\'en daarvan is het Gurobi commercieel softwarepakket, dat de problemen zeer snel kan oplossen. COIN-OR LP\footnote{\href{http://projects.coin-or.org/Clp}{projects.coin-or.org/Clp}}, afgekort Clp, is een open source Linear Programming solver in \cpp . We kiezen voor Clp, omdat deze software open source is, wat voor onze opdrachtgever aantrekkelijker is dan Gurobi, omdat daar wel een licentie voor gekocht moet worden.

\subsection{Werkwijze}
Tijdens de ontwikkelfase gaat het team werken met scrum in wekelijkse sprints. De techniek Scrum wordt beschreven in \emph{The Scrum Guide} \cite{schwaber2011}. Scrum is een framework voor het ontwikkelen van software dat ervoor zorgt dat de opdrachtgever en andere belanghebbenden betrokken zijn bij de implementatie, de gemaakte voortgang en de tot dan toe gemaakte software. Hierdoor kan er ook betere terugkoppeling gegeven worden aan de projectleden, wat zorgt voor een betere samenwerking en een betere vervulling van de wensen van de opdrachtgever. Het ontwikkelingsproces bestaat uit meerdere sprints, die in dit geval elk een week duren. Van tevoren is er besloten op welke taak gefocust zal worden in welke week. Aan het begin van elke sprint wordt er besproken hoe het met de planning staat, wat er gedaan is en wat er in die sprint ge\"implementeerd gaat worden. Deze besluiten worden in elke sprint nog dagelijks bekeken door het houden van scrummeetings van ongeveer een kwartier. Aan het einde van de sprint zal de gemaakte functionaliteit voorgelegd worden aan de opdrachtgever, zodat de feedback gebruikt kan worden voor de volgende sprint. 

De taken die nog gedaan moeten worden, worden in de productbacklog bijgehouden. Deze productbacklog is de lijst met taken die in het project nog gedaan moeten worden. Deze houden we bij met de 'issue tracker' van BitBucket\footnote{\href{http://bitbucket.org}{bitbucket.org}}. Hierbij kan elke taak in \'e\'en van de drie categori\"en en in \'e\'en van de vijf prioriteitsgraden ingedeeld worden. De sprintbacklog is de lijst met taken die in de huidige sprint gedaan worden. Ook deze houden wij bij met de 'issue tracker' van BitBucket, door de taak aan iemand toe te wijzen. Als een taak klaar is, kan deze als 'done' worden gemarkeerd. Een taak is pas klaar als deze goed werkt, er voldoende tests voor zijn en de tests allemaal slagen.

\subsection{Hulpmiddelen}
Als versiebeheersysteem gebruiken we BitBucket. Op deze website zijn private Git repositories gratis beschikbaar. Bij de concurrent GitHub zijn ook private Git repositories beschikbaar, maar dan moet het account wel geactiveerd worden met een TU Delft e-mailadres. Op BitBucket is dit dus eenvoudiger in te stellen. We hebben voor Git gekozen en niet voor Subversion, omdat we meer ervaring hebben met Git. BitBucket heeft ook een ingebouwd issue-trackingsysteem waarvan wij gebruik gaan maken. Dit is in Subversion minder makkelijk in te stellen.

Om ervoor te zorgen dat we altijd werkende code hebben op Bitbucket, gaan we continuous integration gebruiken. Daarvoor is een server waarop Jenkins is ge\"intalleerd erg geschikt. In het geval dat iemand code uploadt die niet werkt, geeft Jenkins daarvan automatisch een melding, zodat wij de code kunnen verbeteren en we weer werkende code hebben op Bitbucket.  

Om er voor te zorgen dat onze code goed onderhoudbaar blijft, gaan we ook een stylechecker gebruiken. De stylechecker \emph{cppcheck} controleert statisch de code op fouten en geeft waarschuwingen wanneer er bijvoorbeeld variabelen worden gedeclareerd die nooit gebruikt worden. De stylechecker \emph{cpplint} controleert op de Google code guideline voor \cpp. Ook \emph{cpplint} is niet perfect, zo zeggen ze zelf, omdat het niet alle fouten ontdekt, maar wij denken dat dit wel een goede toevoeging is. We hebben ook naar de alternatieven \emph{Vera\texttt{++}} en \emph{cxxchecker}, maar deze waren niet gemakkelijk te installeren.
