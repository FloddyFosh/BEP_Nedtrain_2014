\section{User Stories}
\setcounter{userstory}{0}
\nextUserStory
\textbf{Als} gebruiker \textbf{wil} ik op een knop kunnen drukken \textbf{om} de laatste actie(s) ongedaan te maken. Het ongedaan maken van de acties gebeurt van meest recent tot minst recent. 
\begin{itemize} [label=\emph{ - Gegeven}, itemindent=2.5em, labelsep=0.3em]
    \item er is een actie gedaan door de gebruiker nadat de tool is gestart. Het is namelijk niet mogelijk om acties ongedaan te maken in een vorige sessie. 
    \item de actie betreft niet: het openen of sluiten van een instantie.
    \item de actie betreft wel: een aanpassing aan een taak, een project of een resource. 
\end{itemize}

\nextUserStory
\textbf{Als} gebruiker \textbf{wil} ik op een knop kunnen drukken \textbf{om} de laatste undo actie te herstellen. 

\nextUserStory
\textbf{Als} gebruiker \textbf{wil} ik op een knop drukken op een tab van een instantie \textbf{om}, die instantie te sluiten. 

\nextUserStory
\textbf{Als} gebruiker \textbf{wil} ik door middel van het indrukken van de 'shift' toets en scrollen met de muis \textbf{om} horizontaal te scrollen. 
\begin{itemize} [label=\emph{ - Gegeven}, itemindent=2.5em, labelsep=0.3em]
    \item de muis staat in het gebied wat ik, als gebruiker, wil horizontaal wil bewegen. 
\end{itemize}

\nextUserStory
\textbf{Als} gebruiker \textbf{wil} ik door middel van het indrukken van de 'ctrl' toets en scrollen met de muis \textbf{om} te kunnen in- en uitzoomen. 
\begin{itemize} [label=\emph{ - Gegeven}, itemindent=2.5em, labelsep=0.3em]
    \item de muis staat in het gebied wat ik, als gebruiker, wil in- en uitzoomen. 
\end{itemize}

