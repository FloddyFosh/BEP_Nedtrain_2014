\section{User Stories}
\setcounter{userstory}{0}
\nextUserStory
\textbf{Als} gebruiker \textbf{wil} ik op een knop kunnen drukken \textbf{om} de resources-chains weer te geven op het scherm. Die chains worden afgebeeld als lijnen die van taak naar taak gaan. 
\beginGegeven
    \item dat er een probleeminstantie geladen is.
    \item dat de probleeminstantie opgelost is met de solver die chaining gebruikt. 
\endGegeven

\nextUserStory
\textbf{Als} gebruiker \textbf{wil} ik op een knop kunnen drukken \textbf{om} de flexibiliteit van taken zichtbaar te maken. Die flexibilteit wordt weergegeven met een lijn die begint bij de vroegste begintijd en stopt bij de laaste eindtijd.
\beginGegeven
    \item dat er een probleeminstantie geladen is.
    \item dat de probleeminstantie opgelost is met de solver die chaining gebruikt. 
\endGegeven

\nextUserStory
\textbf{Als} gebruiker \textbf{wil} ik op een knop kunnen drukken \textbf{om} de laatste actie ongedaan te maken. \textbf{Als} ik, als gebruiker, nog een keer op diezelfde knop druk, \textbf{wil ik} dat de actie daarvoor ongedaan wordt gemaakt. Het ongedaan maken van de acties gebeurt dus van meest recent tot minst recent. 
\beginGegeven
    \item er is een actie gedaan door de gebruiker nadat de tool is gestart. Het is namelijk niet mogelijk om acties ongedaan te maken in een vorige sessie. 
    \item de actie betreft niet: het openen of sluiten van een instantie.
    \item de actie betreft wel: een aanpassing aan een taak, een project of een resource. 
\endGegeven

\nextUserStory
\textbf{Als} gebruiker \textbf{wil} ik op een knop kunnen drukken \textbf{om} de laatste undo actie te herstellen.
\beginGegeven
    \item ik, als gebruiker, heb een actie ongedaan gemaakt met de 'undo' knop. 
    \item de undo-actie is in deze sessie uitgevoerd. Met andere woorden de tool is na de undo-actie niet opnieuw opgestart. 
\endGegeven

\nextUserStory
\textbf{Als} gebruiker \textbf{wil} ik op een knop drukken op een tab van een instantie \textbf{om} die instantie te sluiten. 
\beginGegeven
    \item er zijn \'e\'en of meerdere instanties geopend.
\endGegeven

\nextUserStory
\textbf{Als} gebruiker \textbf{wil} ik door middel van het indrukken van de 'shift' toets en het scrollen met de muis \textbf{om} horizontaal te scrollen. 
\beginGegeven
    \item de muis staat in het gebied wat ik, als gebruiker, horizontaal wil bewegen.
    \item het gebied wat ik, als gebruiker, horizontaal wil scrollen, heeft een scrollbalk voor horizontaal scrollen. Met andere woorden moet er normaal dus ook de mogelijkheid zijn om horizontaal te scrollen.
\endGegeven

\nextUserStory
\textbf{Als} gebruiker \textbf{wil} ik, de 'ctrl' toets indrukken en tegelijkertijd scrollen met de muis \textbf{om} te kunnen in- en uitzoomen. 
\beginGegeven
    \item de muis staat in het gebied wat ik, als gebruiker, wil in- en uitzoomen. 
    \item het gebied dat in- of uitgezoomd moet worden betreft een timeline met resources en/of taken.
\endGegeven

\nextUserStory
\textbf{Als} gebruiker \textbf{wil} ik, twee activiteiten met elkaar kunnen wisselen van plek op het scherm \textbf{door} op een knop te drukken. 
\beginGegeven
    \item er is bij een activiteit $A$ op de knop naar beneden gedrukt \emph{en} er is op het scherm onder activiteit $A$ nog een andere activiteit $B$, dan worden activiteiten $A$ en $B$ met elkaar gewisseld. Activiteit $A$ staat nu direct onder activiteit $B$.
    \item er is bij een activiteit $A$ op de knop naar boven gedrukt \emph{en} er is op het scherm boven activiteit $A$ nog een andere activiteit $B$, dan worden activiteiten $A$ en $B$ met elkaar gewisseld. Activiteit $A$ staat nu direct boven activiteit $B$.
\endGegeven

\nextUserStory
\textbf{Als} gebruiker \textbf{wil} ik, twee resources met elkaar kunnen wisselen van plek op het scherm \textbf{door} op een knop te drukken.
\beginGegeven
    \item er is bij een resource $A$ op de knop naar beneden gedrukt \emph{en} er is op het scherm onder resource $A$ nog een andere resource $B$, dan worden resources $A$ en $B$ met elkaar gewisseld. Resource $A$ staat nu direct onder resource $B$.
    \item er is bij een resource $A$ op de knop naar boven gedrukt \emph{en} er is op het scherm boven resource $A$ nog een andere resource $B$, dan worden resource $A$ en $B$ met elkaar gewisseld. Resource $A$ staat nu direct boven resource $B$.
\endGegeven

\nextUserStory
\textbf{Als} gebruiker \textbf{wil} ik de flexibiliteit van de oplossing zien op het scherm. De flexibiliteit wordt weergegeven door voor elke taak een flexibiliteitsinterval weer te geven. Dit flexibiliteitsinterval laat zien wanneer een activiteit \emph{mag} beginnen en wanneer het klaar \emph{moet} zijn. Als elke taak binnen het eigen flexibiliteitsinterval gedaan wordt, dan kan elke taak volgens dit schema worden uitgevoerd. Deze flexibiliteitsintervallen worden weergegeven:
\beginGegeven
    \item de optie 'Paint flexibility intervals' aangevinkt staat.
    \item de probleeminstantie is opgelost met een solver dat deze flexibiliteitsintervallen berekend. 
\endGegeven

\nextUserStory
\textbf{Als} gebruiker \textbf{wil} ik na het oplossen van een probleeminstantie de duur van een activiteit kunnen aanpassen d.m.v. met de muis op de rechterkant van een activiteit te verslepen. 
\beginAls
    \item de optie 'Paint flexibility intervals' staat aangevinkt, dan mag de activiteit niet buiten het flexibiliteitsinterval plaatsvinden. 
    \item de optie 'Paint flexibility intervals' niet staat aangevinkt, dan mag de activiteit wel buiten het flexibiliteitsinverval en niet buiten het feasible interval plaatsvinden. 
\endAls

\nextUserStory 
\textbf{Als} gebruiker \textbf{wil} ik na het oplossen van een probleeminstantie de begintijd van een activiteit kunnen aanpassen door een activiteit te verslepen.
\beginAls
    \item de optie 'Paint flexibility intervals' staat aangevinkt, dan mag de activiteit niet buiten het flexibiliteitsinterval plaatsvinden. 
    \item de optie 'Paint flexibility intervals' niet staat aangevinkt, dan mag de activiteit wel buiten het flexibiliteitsinverval en niet buiten het feasible interval plaatsvinden. 
\endAls

\nextUserStory
\textbf{Als} gebruiker \textbf{wil} ik na het oplossen van probleeminstantie de flexibiliteitsintervallen onzichtbaar maken door op een knop te drukken. 
\beginGegeven
    \item er worden flexibiliteitsintervallen weergegeven.
\endGegeven

\nextUserStory
\textbf{Als} gebruiker \textbf{wil} ik dat activiteiten worden verplaats, zodat het begin van de activiteit op hetzelfde moment is als het begin van het flexibiliteitsinterval. 
\beginAls
    \item er op 'solve' geklikt is, waarbij de solver flexibiliteitsintervallen berekend.
    \item 'Paint flexibility intervals' wordt aangevinkt en de probleeminstantie is al opgelost met een solver dat flexibiliteitsintervallen berekend. 
\endAls

