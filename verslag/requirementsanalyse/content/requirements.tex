\section{Requirements}
De hier opgestelde functionele requirements zijn nodig om een goed en volledig product te krijgen, zoals besproken in het Plan van Aanpak. In het Plan van Aanpak worden een aantal features opgesomd en in drie verschillende categorie\"en opgedeeld. De functionele requirements worden hier weer opgedeeld in diezelfde categorie\"en. De niet-functionele requirements worden niet opgedeeld, aangezien de tool daar altijd aan moet voldoen. In deze sectie wordt met de tool gerefereerd naar de 'NedTrain planner'.

\subsection{Functionele Requirements}
\subsubsection*{Hoge Prioriteit}
\begin{enumerate}
    \item De bestaande functionaliteit moet blijven bestaan en wordt dus slechts uitgebreid.
    \item De tool bevat een nieuwe solver die gebruik maakt van de chaining-methode. In oplossingen van deze solver kunnen geen resource conflicten ontstaan door het verschuiven van taken.
    \item Deze nieuwe solver berekent ook door middel van een Linear Programming-solver voor elke taak een interval waarover de taak verschoven kan worden zonder enige gevolgen voor de andere taken te hebben.
    \item In de gebruikersinterface zijn de flexibiliteitsintervallen, die berekend zijn door de Linear Programming-solver, zichtbaar en taken kunnen hierover verschoven worden.
    \item De nieuwe solver lost probleeminstanties op in het bestaande format en geeft de oplossing in een nieuw format, zodat ook de flexibiliteitsintervallen worden meegegeven in de oplossing.
\end{enumerate}

\subsubsection*{Gemiddelde Prioriteit}
\begin{enumerate}[resume]
    \item De tool moet een undo-functionaliteit hebben. Deze undo-functionaliteit maakt het mogelijk om de laatste tien veranderingen aan taken, projecten en resources ongedaan te kunnen maken. Deze acties zijn niet meer ongedaan te maken na het afsluiten van de tool.
    \item De tool bevat een optie om weer te geven tot welke chains de taken horen. 
\end{enumerate}

\subsubsection*{Lage Prioriteit}
\begin{enumerate}[resume]
    \item De sneltoets ctr\plus  scroll zal zorgen voor inzoomen van de tasks en recources time-line. Zoals daarvoor nu ook al de knoppen 'inzoomen' en 'uitzoomen' voor aanwezig zijn. 
    \item De sneltoets shift\plus scroll zal zorgen voor horizontaal scrollen in de verschillende views, waar horizontaal zoomen mogelijk is.
\end{enumerate}

\subsection{Niet-functionele Requirements}
\begin{enumerate}
    \item Het generen van een oplossing mag niet te lang duren. Voor de grootste geleverde probleeminstantie mag dit niet meer dan 5 minuten bedragen.
    \item Gehele project moet gemakkelijk overdraagbaar zijn naar een volgende groep van ontwikkelaars. Voldoende documentatie en testcode speelt hierbij een rol.
    \item De tool is te gebruiken op Windows 7 computer met minstens 200 MB vrije schijfruimte, 2 GB RAM of meer en een processor met 1 GHz of sneller.
    \item De tool is te gebruiken op een computer met een Linux versie, minstens 200 MB vrije schijfruimte, 2 GB RAM of meer en een processor met 1 GHz of sneller. Op deze computer moet het dan wel mogelijk zijn om grafische interfaces te gebruiken, dit is bijvoorbeeld mogelijk op Ubuntu Desktop. De tool is niet geschikt voor bijvoorbeeld Ubuntu Server. 
    \item De tool moet werken op de nieuwste huidige QT versie, welke op dit moment versie QT 5.2.1 is.
    \item De LP-solver waar de nieuwe solver gebruik van maakt moet open source software zijn, zodat dit ook voor commercieel gebruik gratis is. 
\end{enumerate}
