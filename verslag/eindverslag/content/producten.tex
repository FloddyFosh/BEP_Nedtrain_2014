\section{Opgeleverde Producten en Diensten}

\subsection{Solver}

\subsubsection*{Chaining}


\subsection*{Linear Programming}
Om voor elke taak een flexibiliteitsinterval te vinden, moet er, zoals beschreven in paragraaf \ref{subsec:probleemoplossing}, een LP-probleem opgelost worden. Om dit gemakkelijk te kunnen doen, wordt de CLP\footnote{\href{https://projects.coin-or.org/Clp}{projects.coin-or.org/Clp}} (COIN-OR Linear Programming) library gebruikt. Deze library verwacht als input een stelsel lineaire vergelijkingen en een variabele om te optimaliseren, en geeft als output de geoptimaliseerde waarde en voor elke variabele de toegekende waarde.

Hoe het toevoegen van rows gebeurt enzo...

\subsection{Interface}
- weergeven van chains
- weergeven van flexibiliteitsintervals
- verschuiven van taken/resources
- sluiten van instanties in tabjes (grote knop weg)

\subsection{Windows 7}
Om de toegankelijkheid van de applicatie te vergroten, is ervoor gezorgd dat deze ook op het besturingssysteem Windows 7 werkt. Hierbij is het ook nog steeds mogelijk om deze op Unix-based systemen te draaien, zoals Ubuntu. Alle functies van het programma moeten dus zowel op Windows 7 als op Ubuntu correct werken.

\subsection{Qt 5.2}
De bestaande applicatie was ontwikkeld in versie 4.8 van het Qt framework, maar om de applicatie zo up-to-date mogelijk te houden, is deze geport naar Qt versie 5.2. Hierdoor hoeft een eventuele groep die de applicatie verder gaat ontwikkelen, niet met een oude versie van het Qt framework te werken. Deze upgrade betekent echter ook dat de applicatie niet meer werkt met Qt 4, maar alleen met versie 5.