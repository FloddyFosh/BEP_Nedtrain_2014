\section{Probleemanalyse}
\label{sec:probleemanalyse}

Het roosterprobleem waarmee NedTrain te maken heeft, zal in dit hoofdstuk formeel gedefini\"eerd worden. Vervolgens zal een aantal oplosmethoden besproken worden voor de verschillende onderdelen van dit probleem.

\subsection{Resource Constrained Project Scheduling Problem}
Het probleem waar NedTrain mee te maken heeft, staat ook wel bekend als het \emph{Resource Constrained Project Scheduling Problem (RCPSP)}. Dit probleem kan in verschillende variaties en uitbreidingen voorkomen, maar het globale probleem voor elke variatie is het vinden van een rooster bestaande uit activiteiten, zodanig dat de resources die de activiteiten nodig hebben niet de maximale capaciteit van de resources overschrijdt en dat de opgelegde beperkingen niet geschonden worden \cite{lombardi2010constraint}. Het is bekend dat het RCPSP een NP-moeilijk probleem is \cite{blazewicz1983scheduling}. Dat wil zeggen dat er geen polynomiaal algoritme voor dit probleem bekend is, waardoor alleen kleine instanties van dit probleem in een redelijke tijd opgelost kunnen worden. 

De instanties van NedTrain bestaan uit projecten $P = \{p_1, \dots , p_n\}$ die uit \'e\'en of meer subactiviteiten $v_{a,b}$ bestaan. De verzameling van alle activiteiten, $\bigcup_{\{p_a \in P\}} V_a$, wordt gedefinieerd als de verzameling $V = \{v_{1,1}, \dots , v_{n,m}\}$. Een project $p_a$ heeft een starttijd \emph{(release time)} $rs_a$ en een deadline $dl_a$. De subactiviteiten van project $p_a$ mogen pas vanaf $rs_a$ uitgevoerd worden, maar moeten wel uiterlijk op $dl_a$ klaar zijn. De b$^e$ activiteit van project $p_a$ wordt weergegeven als $v_{a,b}$. Deze activiteit heeft een tijdsduur van $d_{a,b}$ tijdseenheden.

Er zijn drie soorten voorwaarden \emph{(constraints)} waar rekening mee gehouden moet worden bij het oplossen van een RCPSP. Dit zijn tijdsconstraints, voorrangsrelaties \emph{(precedence constraints)} en resource constraints. De tijdsconstraints zijn de voorwaarden aan de hierboven genoemde starttijden en deadlines van de projecten. Een eventuele voorrangsrelatie $v_{i,j} \prec v_{u,v}$ kan bestaan tussen twee activiteiten $v_{i,j}$ en $v_{u,v}$, waardoor activiteit $v_{u,v}$ pas uigevoerd mag worden nadat activiteit $v_{i,j}$ voltooid is. Tenslotte zijn oplossingen ook gelimiteerd door de resource constraints. Elke activiteit kan namelijk van verschillende soorten resources $r_k$ een bepaald aantal units nodig hebben. Het aantal units dat activiteit $v_{i,j}$ van resource $r_k$ nodig heeft, wordt genoteerd als $req(v_{i,j},r_k)$.

Een formele definitie van RCPSP kan nu als volgt opgesteld worden:

\textbf{Gegeven:}\\
Een verzameling projecten $P = \{p_1, \dots ,p_n\}$ met voor elk project $p_a$ een starttijd $rs_a$, een deadline $dl_a$ en een verzameling activiteiten $V_a = \{v_{a,1},\dots ,v_{a,m}\}$ met voor elke activiteit $v_{a,b}$ een tijdsduur $d_{a,b}$. Daarnaast een verzameling van voorrangsrelaties $E \subseteq V \times V = \{(v_{a,b},v_{c,d}) | v_{a,b} \in V_a \wedge v_{c,d} \in V_c \wedge p_a,p_c \in P \}$, een verzameling van $w$ resources $R = \{r_1, \dots ,r_w\}$ met voor elke resource $r_i$ een eindige capaciteit $cap(r_i)$, en voor elke activiteit $v_{a,b} \in V$ een hoeveelheid van resource $r_k \in R$ die nodig is, $req(v_{a,b},r_k)$.

\textbf{Vind:}\\
Een doenlijk schema $S = \{s_1,\dots ,s_m\}$ bestaande uit starttijden $s_{a,b}$ voor elke activiteit $v_{a,b} \in V$, zodat $rs_a \leq s_{a,b} \wedge s_{a,b} + d_{a,b} \leq dl_a$, waarbij voor elke voorrangsrelatie $(v_{a,b},v_{c,d}) \in E$ geldt dat $s_{a,b} + d_{a,b} \leq s_{c,d}$. Bovendien mag op geen enkel tijdstip de capaciteit van een resource overschreden worden, wat inhoudt dat $\forall r_i \in R$ en $\forall t \in \mathbb{R}$ geldt dat $\Sigma _{\{v_{a,b} \in V | s_{a,b} \leq t \leq s_{a,b} + d_{a,b}\}} req(v_{a,b},r_i) \leq cap(r_i)$.

\subsection{Flexibiliteit}
\label{subsec:flexprobleem}
Om ervoor te zorgen dat de flexibiliteit van een schema geoptimaliseerd kan worden, moet de flexibiliteit van een schema eerst goed gedefini\"eerd worden. Een mogelijke (na\"ieve) maat voor de flexibiliteit van een activiteit is simpelweg de grootte van het toegelaten interval van deze activiteit. Deze grootte wordt berekend door het verschil te nemen tussen de vroegste starttijd (est) en de laatste starttijd (lst). De totale flexibiliteit van een schema kan vervolgens berekend worden met Formule \ref{eq:flexSlecht}. Deze flexibiliteit wordt echter niet be\"invloed als er voorrangsrelaties tussen taken worden toegevoegd, terwijl dit wel degelijk een negatieve invloed heeft op de flexibiliteit van het schema. Een aantal andere flexibiliteitsmaten, besproken in \cite{wilson2013flexibility}, houdt ook geen rekening met onderlinge afhankelijkheden van taken bij het vaststellen van de flexibiliteit.

\begin{align}
\label{eq:flexSlecht}
    flex_{totaal} = \sum_{v \in V} (est(v) - lst(v))
\end{align}

De flexibiliteitsmaat van Wilson \emph{et al.} \cite{wilson2013flexibility} houdt er echter wel rekening mee dat een schema minder flexibel wordt door het toevoegen van voorrangsrelaties. Deze maat zal dus in het vervolg gebruikt worden om de flexibiliteit van een schema te maximaliseren. Voor het vaststellen hiervan moet voor elke activiteit $v_{a,b}$ een interval $[l_v,u_v]$ berekend worden, zodat elke activiteit in zijn eigen interval verschoven kan worden, zonder dat daardoor andere activiteiten verplaatst hoeven te worden. Dit houdt dus in dat de verzameling van intervallen $I_S = \{I_v = [l_v,u_v] | v \in V\}$ voldoet desda voor elke $v \in V$ en voor elke $t_v \in [l_v,u,v]$, een toewijzing van starttijd $t_v$ aan activiteit $v$ een consistent schema oplevert. Vervolgens kan de flexibiliteit van het schema vastgesteld worden door de lengtes van de intervallen bij elkaar op te tellen, zoals in formule \ref{eq:flex1}:

\begin{align}
\label{eq:flex1}
    flex_{totaal} = \sum_{v \in V} (u_v - l_v)
\end{align}

\subsection{Oplossing}
\label{subsec:probleemoplossing}
In de bestaande software was al een algoritme ge\"implementeerd dat een doenlijk schema kan vinden door middel van het ESTA$^+$ algoritme \cite{ronaldevers2010}.
Dit algoritme lost het RCPSP op door het vinden en oplossen van \emph{contention peaks}, plekken in het resource profiel waar er meer resource units gebruikt worden dan er beschikbaar zijn. Deze worden veroorzaakt doordat er te veel activiteiten die dezelfde resource gebruiken tegelijk uitgevoerd worden. Deze contention peaks kunnen ge\"elimineerd worden door tussen ten minste 1 paar van de betreffende activiteiten een voorrangsrelatie toe te voegen. Het $ESTA^+$ algoritme levert als oplossing voor elke activiteit een starttijd, zodat het gehele schema consistent is. Als er echter met activiteiten geschoven wordt, kan het schema nog wel inconsistent worden.

\subsubsection{Chaining}
\label{subsubsec:chainingoplossing}
Om onafhankelijke flexibiliteitsintervallen te kunnen bepalen, moeten activiteiten vrij verschoven kunnen worden, zonder dat hierdoor het schema inconsistent wordt. Hier kan het chaining algoritme voor zorgen. Dit algoritme splitst elke resource op in zogenaamde units van elk een capaciteit van 1. Vervolgens wordt aan elke resource unit een zogenaamde chain, een gesorteerde lijst van activiteiten, toegekend. De activiteiten in deze chain zullen de enige activiteiten zijn die gebruik mogen maken van de betreffende resource unit. In een chain geldt voor elk paar opeenvolgende activiteiten, $v_{a,b}$ en $v_{c,d}$, dat er een voorrangsrelatie $v_{a,b} \prec v_{c,d}$ bestaat. Hierdoor kunnen twee activiteiten uit dezelfde chain nooit tegelijk uitgevoerd worden. Bovendien wordt een activiteit die meerdere resource units nodig heeft, voor elke resource unit aan een aparte chain toegevoegd. Als een bepaalde activiteit dus bijvoorbeeld drie resource units nodig heeft, is deze activiteit dus aanwezig in precies drie verschillende chains. Op deze manier kan het aantal benodigde resource units nooit groter zijn dan de capaciteit, omdat twee activiteiten uit dezelfde chain nooit tegelijk uitgevoerd kunnen worden en het aantal chains van een resource gelijk is aan de capaciteit daarvan.

De output van het chaining algoritme bestaat uit een zogenaamd \emph{Partial Order Schedule (POS)}, wat een schema is waarvoor geldt dat elk schema dat qua tijdsconstraints consistent is, ook consistent is met de resource constraints. Het exacte chaining algoritme is weergegeven in Algoritme \ref{alg:chaining}.

\begin{algorithm}
\caption{Chaining \cite{policella2007precedence}}
\label{alg:chaining}
\textbf{Input:} Een probleem P en daarvan een (fixed-time) oplossing $S$ \\
\textbf{Output:} Een Partial Order Schedule $POS^{ch}$
\begin{algorithmic}[1]
  \Function{Chaining}{$P, S$}
    \Let{$POS^{ch}$}{$P$}
    \State Sorteer alle activiteiten op hun starttijd in $S$  
    \State Initialiseer alle chains leeg
    \For{\textbf{each} resource $r_j$}
      \For{\textbf{each} activity $v_i$}
        \For{$1$ \textbf{to} $req_{ij}$}
          \Let{$k$}{$SelectChain(v_i,r_j)$}
          \Let{$v_k$}{$last(k)$}
          \State{$AddConstraint(POS^{ch},v_k \prec v_i)$}
          \Let{$last(k)$}{$v_i$}
        \EndFor
      \EndFor
    \EndFor
    \State{\Return{$POS^{ch}$}}
  \EndFunction
\end{algorithmic}
\end{algorithm}

De functie $AddConstraint(P,v_k \prec v_i)$ zorgt ervoor dat de voorrangsrelatie $v_k \prec v_i$ aan het probleem toegevoegd wordt en werkt vervolgens de starttijden van de betrokken activiteiten bij. In de functie $SelectChain(v_i,r_j)$ wordt een chain gekozen waar activiteit $v_i$ aan toegevoegd zal worden. Een simpele, voor de hand liggende methode is om een activiteit gelijk aan een chain toe te voegen als er een geschikte chain gevonden is. Een chain is geschikt voor een activiteit $v_i$ als de eindtijd van het laatste element van de chain, $v_k$, eerder is dan de starttijd van $v_i$, dus $s_k + d_k \leq s_i$. Er kan ook voor gekozen worden om een willekeurige chain te selecteren. Hiervoor moeten eerst alle geschikte chains gevonden worden, en wordt er vervolgens uit deze verzameling een willekeurig element gekozen. Tenslotte is er ook in \cite{policella2004generating} een heuristiek voorgesteld die probeert om het aantal voorrangsrelaties tussen activiteiten uit verschillende chains te minimaliseren. Deze heuristiek, weergegeven in \ref{alg:selectchain1}, probeert eerst een chain $k$ te vinden waarvoor geldt dat er al een voorrangsrelatie $last(k) \prec v_i$ bestaat. Als een dergelijke chain bestaat, wordt de voorrangsrelatie $last(k) \prec v_i$ toegevoegd en anders wordt er voor $v_i$ een willekeurige chain uitgekozen.

Als $v_i$ nog meer units van dezelfde resource nodig heeft, wordt er eerst geprobeerd om $v_i$ aan chains toe te voegen die ook $last(k)$ als laatste element hebben, omdat er dan geen nieuwe voorrangsrelatie toegevoegd hoeft te worden. De voorrangsrelatie $v_i \prec last(k)$ bestaat immers al. Hierna komen pas andere chains, die $last(k)$ niet als laatste element hebben, in aanmerking voor activiteit $v_i$.

\begin{algorithm}
\caption{SelectChain Heuristiek \cite{policella2004generating}}
\label{alg:selectchain1}
\textbf{Input:} Een probleeminstantie $P$, een verzameling voorrangsrelaties $E$, een activiteit $v_i$ en een resource $r_j$.
\begin{algorithmic}[1]
  \Function{SelectChain}{$P, v_i, r_j$}
  	\Let{$C_a$}{$\{$chain $c | last(c) \in E\}$}
  	\If{$C_a \neq \emptyset$}
  		\Let{$k$}{een willekeurige chain uit $C_a$}
  	\Else
  		\Let{$k$}{een willekeurige geschikte chain}
  	\EndIf
    \State{$AddConstraint(P,v_k \prec v_i)$}
    \If{$req(v_i,r_j) > 0$}
    		\Let{$C_k$}{$\{$chain $c | last(c) == last(k)\}$}
    		\Let{$\overline{C_k}$}{$\{$chain $c | last(c) \neq last(k)\}$}
    		\For{$req = req(v_i,r_j)-1 \dots 0$}
    			\If{$C_k \neq \emptyset$}
    				\Let{$c_k$}{willekeurig element in $C_k$}
    				\State{$AddConstraint(P,last(c_k) \prec v_i)$}
    				\Let{$C_k$}{$C_k - c_k$}
    			\Else
    				\Let{$c_k$}{willekeurig element in $\overline{C_k}$}
    				\State{$AddConstraint(P,last(c_k) \prec v_i)$}
    				\Let{$\overline{C_k}$}{$\overline{C_k} - c_k$}
    			\EndIf
    		\EndFor
    	\EndIf
  \EndFunction
\end{algorithmic}
\end{algorithm}

\subsubsection{Linear Programming}
\label{subsubsec:flexoplossing}
Nu alle constraints door het chaining algoritme aan het model zijn toegevoegd, kan de flexibiliteit van het schema worden geoptimaliseerd, door de probleeminstantie om te zetten naar een Linear Programming \emph{(LP)} probleem. Dit probleem kan in polynomiale tijd opgelost worden door middel van een LP-solver.

De flexibiliteit van een taak kan berekend worden door middel van $flex = t^+ - t^-$, waarbij $t^+$ de laatste starttijd (l.s.t.) en $t^-$ de eerste starttijd (e.s.t.) van een taak $t$ is. De flexibiliteit van het hele schema wordt, zoals in paragraaf \ref{subsec:flexprobleem}, gemeten door de som van de flexibiliteit van alle taken $t \in T$ te nemen, zoals in formule \ref{eq:flex} is weergegeven. 

\begin{align}
\label{eq:flex}
    flex_{totaal} = \sum_{t \in T} (t^+ - t^-)
\end{align}

Het doel is de flexibiliteit te maximaliseren en ervoor te zorgen dat er tegelijkertijd wordt voldaan aan de constraints $(t - t' \leq c) \in C$. Ook mag de e.s.t. niet na de l.s.t. zijn ($t^- \leq t^+ \Rightarrow 0 \leq t^+ - t^-$). Hieruit volgt het LP-probleem zoals in formule \ref{eq:flexLP} weergegeven is.

\begin{align}
\label{eq:flexLP}
\begin{aligned}
        \text{max:}& \quad \sum_{t \in T} (t^+ - t^-) & \\
 \text{subject to:}& \quad 0 \leq t^+ - t^- & \forall t \in T \\
                   & \quad t^+ - t^- \leq c & \forall (t^+ - t^{'-} \leq c) \in C
\end{aligned}
\end{align}

Het nadeel van de formulering in formule \ref{eq:flexLP}, is dat de flexibiliteit niet netjes verdeeld hoeft te worden over de taken. Bij het testen is zelfs gebleken dat de 'Simplex' methode meestal de flexibiliteit juist heel slecht verdeelt. Met de 'Interior Point' methode bleek de flexibiliteit wel beter verdeeld te worden. Beide methoden zijn beschikbaar in de CLP library en komen wel uit op hetzelfde antwoord van de geoptimaliseerde variabele, maar verschillen qua waarden van de andere variabelen. Het exacte verschil tussen deze twee methoden ligt buiten het kader van dit verslag.

Met het LP-probleem, zoals in formule \ref{eq:flexfirstLP} staat, wordt de minimale flexibiliteit van elke taak $min$ gemaximaliseerd \cite{wilmer13}. Vervolgens wordt van het LP-probleem, zoals in formule \ref{eq:flexsecondLP}, de totale flexibiliteit gemaximaliseerd. Dit is hetzelfde stelsel als in formule \ref{eq:flexLP}, maar nu met als nieuwe voorwaarde dat elke taak minimaal een flexibiliteit van $min$ moet hebben ($min \leq t^+ - t^ -$). 

\begin{align}
\label{eq:flexfirstLP}
\begin{aligned}
        \text{max:}& \quad min & \\
 \text{subject to:}& \quad min \leq t^+ - t^- & \forall t \in T \\
                   & \quad t^+ - t'^- \leq c & \forall (t - t' \leq c) \in C
\end{aligned}
\end{align}

\begin{align}
\label{eq:flexsecondLP}
\begin{aligned}
        \text{max:}& \quad \sum_{t \in T} (t^+ - t^-) & \\
 \text{subject to:}& \quad min \leq t^+ - t^- & \forall t \in T \\
                   & \quad t^+ - t'^- \leq c & \forall (t - t' \leq c) \in C
\end{aligned}
\end{align}

Door het toevoegen van de nieuwe constraint $min \leq t^+ - t^-$, kan het voorkomen dat de totale flexibiliteit $flex_{totaal}$ lager is dan bij het gebruik van formule \ref{eq:flexLP}, maar in dit geval is de flexibiliteit wel beter over alle activiteiten verdeeld.
