\section{Inleiding}
Dit project betreft het uitbreiden en verbeteren van bestaande planningssoftware. Hierbij gaat het in het bijzonder om het vinden van flexibele schema's, waarin taken gemakkelijk verschoven kunnen worden, zonder dat dit problemen oplevert. Dit alles gebeurt in opdracht van NedTrain en de TU Delft, die deze software willen gebruiken voor zowel onderzoek als educatieve doeleinden. Voor gebruikers biedt het de mogelijkheid schedulingsproblemen op te lossen en te onderzoeken. Ook kan de applicatie de stappen, die het algoritme neemt tijdens het oplossen, grafisch weergeven. 

Het bedrijf NedTrain is een dochterbedrijf van de Nederlandse Spoorwegen (NS), net als bijvoorbeeld NS Reizigers B.V., dat verantwoordelijk is voor het personenvervoer. NedTrain is het dochterbedrijf dat zorgt voor het onderhoud aan alle treinen aangeleverd door NS Reizigers. Dat onderhoud bestaat niet alleen uit het repareren van treinen, maar ook uit het reinigen en moderniseren van treinen. Er moet hierbij gezorgd worden dat van de grofweg 3000 bestaande treinstellen bij NS er op elk moment genoeg beschikbaar zijn om gebruikt te kunnen worden door NS Reizigers. Om dit te bereiken staan er op elk moment van de dag ongeveer 250 treinstellen tegelijk voor onderhoud op meer dan 30 locaties door heel Nederland bij NedTrain. Hier werken in totaal ongeveer 3500 mensen 24 uur per dag, 7 dagen per week aan de opgenoemde taken. Het hoofdkantoor van NedTrain staat in Utrecht.

Ten eerste zal in dit eindverslag gedefini\"eerd worden wat het probleem van de opdrachtgevers is, waar tijdens dit project een oplossing voor gezocht zal worden. Vervolgens zal het hoofdstuk probleemanalyse gericht zijn op het analyseren en oplossen van het schedulingsproblemen waar de NedTrain planner mee te maken krijgt. Daarna zullen in het hoofdstuk Ontwerp de ontwerpkeuzes genoemd en toegelicht worden, en worden in het daarop volgende hoofdstuk alle opgeleverde producten en diensten genoemd. Omdat het ook belangrijk is dat het schedulingsalgoritme snel uitgevoerd wordt en goede oplossingen levert, zal er ook een hoofdstuk gewijd zijn aan de performance van de solver. Vervolgens zal besproken worden hoe het ontwikkelproces gelopen is en wat voor hulpmiddelen er gebruikt zijn. Tenslotte zullen er functionaliteiten aanbevolen worden die eventueel later nog ge\"implementeerd kunnen worden, en wordt in de conclusie teruggekeken op het project en wordt een conclusie getrokken met betrekking tot de eisen van project.

Als voorbereiding op dit project is in de eerste weken een aantal documenten opgesteld, namelijk een Plan van Aanpak \emph{(zie bijlage \ref{app:B})}, een Ori\"entatieverslag \emph{(zie bijlage \ref{app:C})} en een Requirementsanalyse \emph{(zie \ref{app:D})}.