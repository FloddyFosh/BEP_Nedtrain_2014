\section{Inleiding}
Dit project betreft het implementeren van het chaining algoritme in combinatie met een LP solver voor een bestaande grafische schedulingtool van het bedrijf NedTrain, de NedTrain planner. Deze oplossing moet op een intu\"itieve manier gevisualiseerd worden door het uitbreiden en verbeteren van de bestaande software. Hierdoor kunnen gebruikers de flexibiliteit van complexe schedulingsproblemen oplossen en onderzoeken.

Ten eerste zal in dit eindverslag het probleem gedefini\"eerd worden, waar tijdens dit project op gefocust wordt. Vervolgens zal het hoofdstuk probleemanalyse gericht zijn op het analyseren en oplossen van het schedulingsprobleem dat opgelost moet worden door de NedTrain planner. Daarna zullen in het hoofdstuk Ontwerp de ontwerpkeuzes genoemd en toegelicht worden, en in het daarop volgende hoofdstuk alle opgeleverde producten en diensten genoemd. Omdat het ook belangrijk is dat het schedulingsalgoritme snel uitgevoerd wordt en goede oplossingen levert, zal er ook een hoofdstuk gewijd zijn aan de performance van de solver. Vervolgens zal besproken worden hoe het ontwikkelproces gelopen is en wat voor hulpmiddelen er gebruikt zijn. Tenslotte zullen er functionaliteiten aanbevolen worden die eventueel later nog ge\"implementeerd kunnen worden, en wordt in de conclusie teruggekeken op het project en wordt een conclusie getrokken met betrekking tot de eisen van project.

Als voorbereiding op dit project zijn in de eerste weken een aantal documenten opgesteld, namelijk een Plan van Aanpak, zie bijlage \ref{app:B}, een Ori\"entatieverslag, zie bijlage \ref{app:C} en een Requirementsanalyse, zie \ref{app:D}.