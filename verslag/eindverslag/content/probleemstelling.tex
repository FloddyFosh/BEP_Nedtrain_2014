\section{Probleemstelling}
\label{sec:probleemstelling}

De NedTrain planner, ontwikkeld in samenwerking met de TU Delft, is een grafische tool die moeilijke schedulingsproblemen kan visualiseren. Met behulp van een zelfgemaakte solver kunnen deze problemen opgelost worden.
De tool geeft vervolgens de mogelijkheid om deze oplossingen te bekijken en aanpassingen te maken om zo de veranderingen te onderzoeken. Deze planner wordt gebruikt door NedTrain om onderzoek te doen naar schedulingsalgoritmen en het verbeteren daarvan.
Daarnaast heeft het voor de TU Delft ook een educatief doel, doordat deze op een visuele manier duidelijke maakt hoe schedulingsalgoritmen te werk gaan. De huidige versie is een resultaat van meerdere projecten van studenten aan de TU Delft. De applicatie dient als proof of concept om te laten zien wat er mogelijk is met complexe algoritmen op het gebied van het plannen van treinonderhoud.
NedTrain zou graag willen dat de laatste ontwikkelingen op het gebied van schedulingsalgoritmen ook in de tool worden ge\"implementeerd. Hierdoor kan NedTrain beter inzicht krijgen in wat voor ontwikkelingen de laatste jaren gemaakt zijn en of ze gebruikt kunnen worden voor andere toekomstige systemen.