\section{Probleemstelling}
\label{sec:probleemstelling}

Het doel van dit project is het verder ontwikkelen van de NedTrain planner, die eerder ontwikkeld is door NedTrain in samenwerking met de TU Delft. Dit is een grafische tool die moeilijke schedulingsproblemen kan oplossen en visualiseren. Het oplossen van probleeminstanties gebeurt door middel van een zogenaamde solver, die onafhankelijk is van de gebruikersinterface. De gebruikersinterface geeft vervolgens de mogelijkheid om deze oplossingen te bekijken en aanpassingen te maken om zo de veranderingen te onderzoeken. De NedTrain planner wordt gebruikt door NedTrain om onderzoek te doen naar schedulingsalgoritmen en het verbeteren daarvan.

NedTrain zou graag willen dat de NedTrain planner wordt uitgebreid met functionaliteit die er voor zorgt dat er flexibele schema's geconstrueerd worden en de flexibiliteit hiervan op een bepaalde manier gemeten wordt. Hierdoor kan NedTrain beter inzicht krijgen in wat voor ontwikkelingen de laatste jaren gemaakt zijn en of ze gebruikt kunnen worden voor andere toekomstige systemen. 

Daarnaast heeft het project voor de TU Delft ook een educatief doel, doordat de software op een visuele manier duidelijk maakt hoe schedulingsalgoritmen te werk gaan. De huidige versie is een resultaat van meerdere projecten van studenten aan de TU Delft. De applicatie dient als proof of concept om te laten zien wat er mogelijk is met complexe algoritmen op het gebied van het plannen van treinonderhoud.