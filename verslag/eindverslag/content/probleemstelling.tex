\section{Probleemstelling}
\label{sec:probleemstelling}

De NedTrain Planner is een uitgebreide grafische applicatie waarin roosterproblemen kunnen worden aangemaakt, ge\"importeerd, gevisualiseerd en aangepast. Door middel van een zogenaamde solver, die onafhankelijk is van de gebruikersinterface en die ingeladen moet worden in de applicatie, kunnen deze roosterproblemen opgelost worden. Vervolgens kan er informatie verkregen en opgevraagd worden over de gevonden oplossing en kan er stapsgewijs door het proces van de solver heen gelopen worden om zo een beter beeld te krijgen van hoe de solver werkt. Daarnaast geeft de NedTrain Planner ook de mogelijk om na het oplossen aanpassingen te maken en de instantie vervolgens opnieuw op te laten lossen. Zo kunnen veranderingen in het resultaat makkelijk weergegeven worden.

E\'en van de mogelijkheden die de NedTrain Planner biedt na het oplossen, is het verschuiven van activiteiten over hun uitvoerbare intervallen. Deze intervallen tonen aan in hoeverre activiteiten verschoven kunnen worden, zonder dat dit leidt tot een inconsistente oplossing. De huidige solver biedt echter alleen de garantie dat het gegenereerde schema een consistente oplossing  van het probleem is. Het is dus mogelijk dat er bij het verschuiven van een activiteit de capaciteit van een resource overschreden wordt, waardoor het schema inconsistent wordt.

De opdrachtgevers hebben, als beschrijving van dit Bachelorproject, een document \emph{(zie Bijlage \ref{app:A})} opgesteld waarin wordt beschreven wat de opdracht van het project is. Na nader overleg tussen het ontwikkelteam en de opdrachtgevers kwam daar een concreet probleem uit. Dit probleem is op te delen in drie gedeeltes. 

Het eerste gedeelte van het probleem is het implementeren van een algoritme dat garandeert dat voor elke tijdsbepaling van de activiteiten dit niet een inconsistent rooster geeft. Er moet dus uiteindelijk met de activiteiten over hun uitvoerbare intervallen verschoven kunnen worden, zonder dat er een resource zijn capaciteit overschrijdt of dat er een andere beperking geschonden wordt.

Het tweede gedeelte van het probleem is om de NedTrain Planner uit te breiden met functionaliteit die er voor zorgt dat er flexibele roosters geconstrueerd en gevisualiseerd kunnen worden. Dit betekent dat activiteiten vrij over een interval moeten kunnen bewegen zonder dat daarbij de starttijd of vorm van een andere activiteit veranderd wordt. Dit geeft NedTrain de mogelijkheid om bij het uitvoeren van een rooster makkelijk aanpassingen te kunnen maken, zonder daarbij het hele rooster hoeven te veranderen of opnieuw te berekenen.

Het laatste gedeelte van het probleem is om informatie te geven over de flexibiliteit en het berekenen hiervan. Daarnaast moeten veranderingen in deze informatie ge\"evalueerd kunnen worden na het geven van gebruikersfeedback. Dit kan door de TU Delft gebruikt worden om beter te begrijpen hoe de algoritmes zich gedragen en kan het helpen om verder onderzoek naar het oplossen van roosterproblemen te stimuleren.

