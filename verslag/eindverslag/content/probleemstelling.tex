\section{Probleemstelling}
\label{sec:probleemstelling}

De NedTrain Planner is een uitgebreide grafische applicatie waarin roosterproblemen kunnen worden aangemaakt, ge\"importeerd, gevisualiseerd en aangepast. Door middel van een zogenaamde solver, die onafhankelijk is van de gebruikersinterface en die ingeladen moet worden in de applicatie, kunnen deze roosterproblemen opgelost worden. Vervolgens kan er informatie verkregen en opgevraagd worden over de gevonden oplossing en kan er stapsgewijs door het proces van de solver heen gelopen worden om zo een betere overzicht te krijgen over hoe de solver werkt. Daarnaast geeft de NedTrain Planner ook de mogelijk om na het oplossen aanpassingen te maken en de instantie  vervolgens opnieuw op te laten lossen. Zo kunnen veranderingen in het resultaat makkelijke weergegeven worden.

E\'en van de mogelijkheden, die de NedTrain Planner biedt na het oplossen, is het verschuiven van activiteiten over hun uitvoerbare intervallen. Deze intervallen tonen aan in hoeverre activiteiten verschoven kunnen worden, zonder dat dit leidt tot een inconsistente oplossing. Echter biedt de huidige solver alleen garantie dat als alle activiteiten op hun vroegste starttijd staan, dat dit een haalbare oplossing is voor het gegeven probleem. Het is dus mogelijk dat er bij het verschuiven van een activiteit een capaciteit van een resource overschreden wordt.

De opdrachtgevers hebben, voor een eerste gesprek, een document \emph{(zie Bijlage \ref{app:A})} opgesteld waarin wordt beschreven wat het project inhoud. Na nader overleg dus het ontwikkelteam en de opdrachtgevers kwamen daar een concreet probleem uit. Dit probleem is op te delen in drie gedeeltes. 

Het eerste gedeelte van het probleem is om een algoritme te implementeren die garandeert dat voor elke tijdsbepaling van de activiteiten dit niet een inconsistent schema geeft. Er moet dus uiteindelijk met de activiteiten over hun uitvoerbare intervallen verschoven kunnen worden, zonder dat er een resource zijn capaciteit overschreidt of dat er een beperking geschonden wordt.

Het tweede gedeelte van het probleem is om de NedTrain Planner uit te breiden met functionaliteit die ervoor zorgt dat er flexibele schema's geconstrueerd en gevisualiseerd kunnen worden. Dit betekend dat activiteiten vrij over een interval moeten kunnen bewegen zonder dat daarbij de starttijd of vorm van een andere activiteit veranderd wordt. Dit geeft NedTrain de mogelijkheid om bij het uitvoeren van een schema makkelijk aanpassingen te kunnen maken, zonder daarbij het hele schema moeten te veranderen of opnieuw te berekenen.

Het laatste gedeelte van het probleem is om informatie te geven over de flexibiliteit en het berekenen hiervan. Daarnaast moeten veranderingen in deze informatie ge\"evalueerd kunnen worden na het geven van gebruikersfeedback. Dit kan door de TU Delft gebruikt worden om beter te begrijpen hoe de algoritmes zich gedragen en kan het helpen om verdere onderzoek naar het oplossen van roosterproblemen te stimuleren.

