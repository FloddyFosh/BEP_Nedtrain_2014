\section{Probleemstelling}
\label{sec:probleemstelling}

Het doel van dit project is het verder ontwikkelen van de NedTrain planner, die eerder ontwikkeld is in samenwerking met de TU Delft. Dit is een grafische tool die moeilijke schedulingsproblemen kan oplossen en visualiseren. Het oplossen van probleeminstanties gebeurt door middel van een zogenaamde solver, die onafhankelijk is van de gebruikersinterface. De gebruikersinterface geeft vervolgens de mogelijkheid om deze oplossingen te bekijken en aanpassingen te maken om zo de veranderingen te onderzoeken. De NedTrain planner wordt gebruikt door NedTrain om onderzoek te doen naar schedulingsalgoritmen en het verbeteren daarvan.

Daarnaast heeft het voor de TU Delft ook een educatief doel, doordat deze op een visuele manier duidelijke maakt hoe schedulingsalgoritmen te werk gaan. De huidige versie is een resultaat van meerdere projecten van studenten aan de TU Delft. De applicatie dient als proof of concept om te laten zien wat er mogelijk is met complexe algoritmen op het gebied van het plannen van treinonderhoud.
NedTrain zou graag willen dat de laatste ontwikkelingen op het gebied van schedulingsalgoritmen ook in de tool worden ge\"implementeerd. Hierdoor kan NedTrain beter inzicht krijgen in wat voor ontwikkelingen de laatste jaren gemaakt zijn en of ze gebruikt kunnen worden voor andere toekomstige systemen.

Om wat meer inzicht te geven in het bedrijf NedTrain, volgt hier waar NedTrain verantwoordelijk voor is en hoe dit georganiseerd is.
De Nederlandse Spoorwegen heeft verschillende dochterbedrijven met elk hun eigen taak. Het bekendste dochterbedrijf is NS Reizigers B.V., dat verantwoordelijk is voor het personenvervoer. NedTrain is het dochterbedrijf dat zorgt voor het onderhoud aan alle treinen aangeleverd door NS Reizigers. Dat onderhoud bestaat niet alleen uit het repareren van treinen, maar ook uit het reinigen en moderniseren van treinen. Er moet hierbij gezorgd worden dat van de grofweg 3000 bestaande treinstellen bij NS er op elk moment genoeg beschikbaar zijn om gebruikt te kunnen worden door NS Reizigers. Om dit te bereiken staan er op elk moment van de dag ongeveer 250 treinstellen tegelijk voor onderhoud op meer dan 30 locaties door heel Nederland bij NedTrain. Hier werken in totaal ongeveer 3500 mensen 24 uur per dag, 7 dagen per week aan de opgenoemde taken. Het hoofdkantoor van NedTrain staat in Utrecht.