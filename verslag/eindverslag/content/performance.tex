\section{Performance}
In dit hoofdstuk een analyse van de looptijd van de solver. Bij deze analyse delen we de solver op in vijf belangrijke delen: parsen, STJN, ESTA$^+$, chaining en LP. Om de looptijd te bepalen gebruiken we een grote probleeminstantie genaamd 'xxl-instance'. Deze probleeminstantie bestaat uit:
\begin{itemize}
    \item 10 resources, met elk 1 resource unit
    \item 180 projecten, met elk tussen de 4 en de 6 activiteiten
    \item 912 activiteiten, die elk gebruik maken van tussen de 1 en 5 resources
\end{itemize}

In tabel [??] en [??] is te zien wat de loop tijden zijn van de onderdelen van de solver met en zonder printen van output. De solver stuurt de output naar \texttt{stdout} Als er wel output wordt geprint duurt het oplossen van de 'xxl-instance' ongeveer $110$ seconden en als er geen output wordt geprint duurt het oplossen ongeveer $1,4$ seconden. Er valt veel snelheidswinst te behalen door het printen te optimaliseren. Als er slim wordt omgegaan wat er wel en niet wordt geprint kan dit al veel tijd schelen. Een mogelijkheid is om niet telkens alle informatie van een nieuw frame te printen, maar alleen het verschil met het vorige frame. Een andere mogelijkheid is om niet met een apparte solver te werken die wordt aangeroepen door de NedTrain Planner, maar de solver te integreren in de NedTrain Planner. Dan hoeft er niets te worden uitgeprint, maar dit heeft wel als nadeel dat er dan niet gemakkelijk een nieuwe solver toegevoegd kan worden aan de NedTrain Planner. Printen naar een tijdelijk bestand in plaats van \texttt{stdout} zou ook een mogelijke versnelling kunnen opleveren. 

\begin{table}[H]
\parbox{.45\linewidth}{
    \centering
    \label{tbl:xxl+output}
    \begin{tabular}{| l | r | r |}
        \hline
                    & Looptijd ($s$)  & Percentage ($\%$) \\
        \hline
        Parsing     & $0.012629$      & $0.0$   \\
        STJN        & $0.002536$      & $0.0$   \\
        ESTA$^+$    & $109.642634$    & $99.2$  \\
        Chaining    & $0.587552$      & $0.5$   \\
        LP          & $0.313129$      & $0.3$   \\
        \hline \hline
        Totaal      & $110.559147$    & $100.0$ \\
        \hline
    \end{tabular}
    \caption{xxl-instance met output}
}
\hfill
\parbox{.45\linewidth}{
    \centering
    \label{tbl:xxl-output}
    \begin{tabular}{| l | r | r |}
        \hline
                    & Looptijd ($s$)& Percentage ($\%$) \\
        \hline
        Parsing     & $0.012075$      & $0.9$   \\
        STJN        & $0.002895$      & $0.2$   \\
        ESTA$^+$    & $1.083015$      & $76.8$  \\
        Chaining    & $0.034146$      & $2.4$   \\
        LP          & $0.277762$      & $19.7$  \\
        \hline \hline
        Totaal      & $1.410065$      & $100.0$ \\
        \hline
    \end{tabular}
    \caption{xxl-instance zonder output}
}
\end{table}
