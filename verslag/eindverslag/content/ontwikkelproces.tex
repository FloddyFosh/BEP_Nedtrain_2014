\section{Ontwikkel Proces}

\subsection{Sprint 1}
Het doel van de eerste sprint was om chaining te implementeren. Tijdens de onderzoeksfase hebben we gezien dat het beschikbaar maken voor Windows 7, waarschijnlijk goed te doen is in de tijd die we daarvoor hebben. Het beschikbaar maken voor Windows 7, was daarom ook een doel in de eerste sprint. 

\subsubsection*{Terugblik}
We zijn er tevreden over het verloop van de eerste sprint. Het porten naar Windows 7 liep met een paar kleine tegenvallers redelijk vlot. Naast het poorten naar Windows 7 is het ook in deze sprint gelukt om de versie van Qt te upgraden naar versie 5.2.1. De basis van ons chaining algoritme werkt, maar de samenwerking tussen GUI en Solver loopt nog niet zoals gewenst. Er is zelf ook al een begin gemaakt aan de LP Solver, een doel dat voor de twee sprint geplant stond. 

\subsection{Sprint 2}
Het voornaamste doel van de tweede sprint is de COIN-OR LP solver te laten samenwerken met onze applicatie. Daarnaast is er gewerkt aan het laten zien van aparte chains met daarbij welke resources er door deze chain gebruikt worden. 

\subsubsection*{Terugblik}
In deze sprint hebben we een aantal tegenvallers gehad. Zo hadden we last van een memory leak, wat redelijk veel tijd kostte om op te lossen. Bovendien moest de bestaande architectuur voor het laten zien van chains vrij veel aangepast worden. Een tweede tegenvaller was het updaten van de software op de jenkins-server. Aangezien we de tools hebben geupgrade naar Qt versie 5.x, moet op de server Qt ook worden geupgrade. Dit ging niet meteen de eerste keer goed, waarbij uiteindeijk is overgegaan tot een volledige herinstallatie. Dit heeft als gevolg dat we een deel van de eerste commits missen in de continious integration grafiek.

\subsection{Sprint 3}
In dze sprint gaan we een demo doen bij onze opdrachtgevers. Hiermee kunnen we controleren of wat wij hebben gemaakt de opdrachtgevers ook echt willen, we kunnen dan nog het \'e\'en en ander aanpassen in deze en volgende sprints. 

\subsubsection*{Terugblik}
Tijdens de demo zijn er nieuwe ide\"en bijgekomen. Ook hebben we afgesproken om de undo/redo-functionaliteit niet te implementeren, aagezien dit veel tijd kost en weinig oplevert. 

\subsection{Sprint 4}

\subsubsection*{Terugblik}

\subsection{Sprint 5}

\subsubsection*{Terugblik}

