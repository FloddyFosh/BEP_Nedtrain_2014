\begin{appendices}
\cfoot[\fancyplain{}{A - 1}]{\fancyplain{}{A - 1}}

\section{Opdrachtomschrijving} \label{app:A}
\subsection*{Project description}
NedTrain has developed a scheduling tool for operational maintenance. This tool consists of a scheduler and an extensive user interface. The proposed project consists in extending the current solver to construct flexible schedules for maintenance jobs. The proposed activities are.

\begin{enumerate}
	\item The addition of constraint posting techniques that provide a set of temporal constraints ensuring the satisfaction of all resource constraints.
	\item The use of an LP solver to measure the flexibility of the resulting constraint system.
	\item The extension of the user interface to provide the user with flexibility information and to evaluate the changes in flexibility after the users feedback.
\end{enumerate}

\subsection*{Company description}

NedTrain is a part of NS Group, which is the largest railway operator in the Netherlands. NS Group can be decomposed in three important parts: railway station exploitation, train maintenance and of course passenger transportation. NedTrain, which performs rolling stock maintenance, is currently contracted to perform maintenance for the rolling stock used by nsr and NS HiSpeed. The fleet used by these two divisions has a size of around 3000 passenger carriages, of which some 250 are under some form of maintenance at any one time. Maintenance operations are spread out over 37 locations throughout the Netherlands. The core task of NedTrain is guaranteeing fleet availability in the broadest sense of the word. Firstly, this means that trains should have as few breakdowns during service as possible. To ensure this, NedTrain has to focus on the quality of their maintenance work.

\subsection*{Auxiliary information}

This information has been provided by me (Cees Witteveen) after consulting Bob Huisman MSc, the director of research at NedTrain Company: bob.huisman@nedtrain.nl
\newpage

\cfoot[\fancyplain{}{\thepage}]{\fancyplain{}{\thepage}}

\section{Plan van Aanpak} \label{app:B}
\newpage
\includepdf[pages=-]{../plan_van_aanpak/verslag.pdf}

\section{Ori\"entatieverslag} \label{app:C}
\newpage
\includepdf[pages=-]{../orientatieverslag/verslag.pdf}

\section{Requirementsanalyse} \label{app:D}
\newpage
\includepdf[pages=-]{../requirementsanalyse/verslag.pdf}

\end{appendices}
