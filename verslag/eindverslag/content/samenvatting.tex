\section*{Samenvatting}
Het bedrijf NedTrain, de opdrachtgever van dit project, beschikt over software om roosterproblemen op te lossen. NedTrain wil deze software echter graag laten uitbreiden met de functionaliteit om flexibele roosters te berekenen. Het implementeren van deze functionaliteit is het hoofdzakelijke doel van dit project. Het probleem dat hier opgelost moet worden, heeft te maken met het probleem dat ook wel bekend staat als het \emph{Resource Constrained Project Scheduling Problem}. Hierbij wordt er naar een rooster gezocht voor bijvoorbeeld het onderhoud aan treinen. Doordat elke trein binnen een bepaalde tijd gerepareerd moet worden en er ook rekening gehouden moet worden met de beschikbare resources, is dit probleem moeilijk om op te lossen.

De bestaande software van NedTrain bestaat uit een interface, genaamd de NedTrain Planner, en een solver, die er voor zorgt dat instanties opgelost worden. Deze software kan echter alleen vaste roosters genereren. Op het moment dat er activiteiten verplaatst worden, kunnen er conflicten ontstaan, waardoor het rooster niet meer geldig is. Dit wordt opgelost door het implementeren van een nieuwe solver. Deze solver maakt gebruik van het chaining algoritme en de \emph{COIN Linear Programming} library. Het chaining algoritme is nodig om te zorgen dat bij het verplaatsen van activiteiten het rooster consistent blijft qua resources. De COIN LP library is gebruikt om flexibiliteitsintervallen te kunnen berekenen, die vervolgens worden weergegeven in de interface. 

Tijdens dit project is gebruik gemaakt van Scrum en zes wekelijkse sprints. Als versiebeheersysteem werd Git gebruikt op een private repository gehost door \href{http://bitbucket.com}{Bitbucket.com}. Als continuous integration systeem is Jenkins gebruikt, maar dit was helaas geen succes. De codekwaliteit is gemeten door de Software Improvement Group, die vervolgens feedback heeft gegeven over verschillende aspecten van de code. 

Er zijn een viertal aanbevelingen voor projecten die de NedTrain Planner gaan doorontwikkelen. Er wordt aanbevolen om een undo-functionaliteit te maken, de communicatie tussen interface en solver sneller te maken, de kwaliteit van de code te verbeteren en activiteiten beter te laten verdelen over alle beschikbare chains.

Van alle wensen van de opdrachtgever zijn alle features met een hoge prioriteit ge\"implementeerd en zijn ook een groot aantal features met middelmatige of lage prioriteit ge\"implementeerd. Er kan dus geconcludeerd worden dat er aan de eisen van de opdrachtgever voldaan is.
