\section{Conclusie}
Achteraf gezien kunnen we een aantal conclusies trekken over dit project. In het Plan van Aanpak (Bijlage \ref{app:B}) wordt een aantal op te leveren features genoemd, onderverdeeld in drie categorie\"en van prioriteit. Alle taken met hoge prioriteit zijn volbracht en in de categorie met taken van gemiddelde prioriteit stond de undo-functionaliteit, die wij niet hebben kunnen implementeren door een gebrek aan tijd. We hebben onderzocht wat er nodig zou zijn om deze functionaliteit te \"implementeren en kwamen tot de conlusie dat dit erg veel tijd zou gaan kosten en niet in verhouding zou staan met wat het zou opleveren. In de categorie lage prioriteit hebben we alles kunnen doen, behalve \emph{"Het verduidelijken van het vergelijken van verschillende oplossingen."} We zijn erg tevreden nu we kunnen concluderen dat we zoveel features hebben kunnen maken.

We hebben allen veel geleerd van de programmeertaal \cpp\ en het Qt framework. Deze waren beide voor ons erg onbekend en vooral \cpp\ leverde soms aardig wat hobbels op in de weg naar succes. 

De samenwerking liep gesmeerd, dit mede doordat wij ook al aan een eerder project hebben samengewerkt. De feedback van onze begeleiders heeft ook zeker bijgedragen aan een beter eindproduct. Het feit dat andere groepen al aan dit project hebben gewerkt en zeer waarschijnlijk ook wij opgevolgd zullen worden door enthousiaste informatici maakt dit een erg realistisch project. Wij vonden het een zeer leerzaam en geslaagd Bachelorproject en zijn tevreden met de resultaten.
