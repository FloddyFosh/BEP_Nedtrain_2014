\section{Conclusie}
Achteraf gezien kunnen we een aantal conclusies trekken over dit project. In het Plan van Aanpak worden een aantal op te leveren features genoemd, verdeeld in drie categori\"en van prioriteit. Alle hoge prioriteit taken zijn volbracht. In de categorie met taken van gemiddelde prioriteit stond de undo-functionaliteit, deze hebben wij niet kunnen implementeren door een gebrek aan tijd. We hebben onderzocht wat er nodig zou om deze functionaliteit te \"implementeren en kwamen tot de conlusie dat dit erg veel tijd zou gaan kosten en niet in verhouding staat met wat het zou opleveren. In de categorie lage prioriteit hebben we alles kunnen doen, behalve \emph{"Het verduidelijken van het vergelijken van verschillende oplossingen."} We zijn erg tevreden nu we kunnen concluderen dat we zoveel features hebben kunnen maken.

We hebben allen veel geleerd van de programmeertaal \cpp\ en het Qt framework. Deze waren beide voor ons erg onbekend, vooral \cpp\ leverde soms aardig wat hobbels op in de weg naar succes. 

De samenwerking liep gesmeerd, dit mede doordat wij ook al aan een eerder project hebben samengewerkt. De feedback van onze begeleiders heeft ook zeker bijgedragen aan een beter eindproduct. Het feit dat andere groepen al aan dit project hebben gewerkt en zeer waarschijnlijk ook wij opgevolgd zullen worden door enthousiaste informatici maakt dit project realistischer dan elk ander project tijdens de bacheloropleiding. Wij vonden dit allen een zeer leerzaam en geslaagd Bacheloreindproject.
