% TU Delft Beamer template
% Author: Maarten Abbink
% Delft University of Technology
% March 2014
% Version 2.0
% Based on original version 1.0 of Carl Schneider
\documentclass{beamer}
\usepackage[dutch]{babel}
\usepackage{calc}
\usepackage[absolute,overlay]{textpos}
\usepackage{amsmath}
\usepackage{amsthm}
\usepackage{graphicx}
\mode<presentation>{\usetheme{tud}}

\title[NedTrain Planner]{NedTrain Planner}
%\subtitle
\institute[TU Delft]{Technische Universiteit Delft}
\author{Chris Bakker, Anton Bouter, Martijn den Hoedt}
\date{2 juli 2014}

\theoremstyle{definition}
\newtheorem{definitie}{Definitie}[section]

% Insert frame before each subsection (requires 2 latex runs)
\AtBeginSubsection[] {
	\begin{frame}<beamer>\frametitle{\titleSubsec}
		\tableofcontents[currentsection,currentsubsection]  % Generation of the Table of Contents
	\end{frame}
}
% Define the title of each inserted pre-subsection frame
\newcommand*\titleSubsec{Next Subsection}
% Define the title of the "Table of Contents" frame
\newcommand*\titleTOC{Outline}

% define a symbol which can be removed if you don't need it
\newcommand{\field}[1]{\mathbb{#1}}
\newcommand{\Zset}{\field{Z}}

\begin{document}

{
% remove the next line if you don't want a background image
\usebackgroundtemplate{\includegraphics[width=\paperwidth,height=\paperheight]{images/background-titlepage.jpg}}%
\setbeamertemplate{footline}{\usebeamertemplate*{minimal footline}}
\frame{\titlepage}
}

\begin{frame}\frametitle{Opdrachtgevers}
\begin{columns}[T] % align columns
    \begin{column}{.55\textwidth}
        \begin{itemize}
            \item NedTrain 
            \begin{itemize}
                \item Nederlandse Spoorwegen
                \item $250$ treinen op $30$+ locaties
                \item $3500$ werknemers 24/7 
                \item ir. Bob Huisman
            \end{itemize}
        \end{itemize}
        \vspace{1.8cm}
        \begin{itemize}
            \item TU Delft
            \begin{itemize}
                \item Algoritmiek groep
                \item prof. dr. Cees Witteveen
            \end{itemize}  
        \end{itemize}
    \end{column}%
    \begin{column}{.45\textwidth}
        \includegraphics[width=4.5cm]{images/logo-nedtrain.jpg}
        \vspace{1cm}
        \includegraphics[width=5cm]{images/tudelft_logo.pdf}
    \end{column}%
\end{columns}
\end{frame}


\begin{frame}\frametitle{Linear Programming}
    \begin{definitie}
        \begin{align}
            \text{max:}& \quad \sum_{t \in T} (t^+ - t^-) & \nonumber \\
            \text{met voorwaarden:} & \quad 0 \leq t^+ - t^- & \forall t \in T \nonumber \\
                                    & \quad t^+ - t^- \leq c & \forall (t^+ - t^{'-} \leq c) \in C \nonumber
        \end{align}
    \end{definitie}
\end{frame}

\begin{frame}\frametitle{Demo}
    \huge{\hfill Tijd voor een demo! \hfill}
\end{frame}


\end{document}
